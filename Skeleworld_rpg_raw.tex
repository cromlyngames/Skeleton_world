
\section{Skeleton World - the (example) game.}



\subsection{Decisions:}

\smallcaps{Scale:} Simplicity. I want a simple game, scaled back where I can. \sidenote{This is because it is an example, so the less I have to write the better as it will make the decisions and structure of the example clear. There is also a thematic unity between playing as skeletons and a bare-bones game}

\smallcaps{Key Imagery:} Coffins `poofing' into existence in the sky and falling into a bone white desert in cloud of dust. A sword wielding skeleton emerging from the ground; cartoon skeleton's in a danse macabre (or a grim fandango); a burning skeleton throwing fireballs,  armour made from hundreds of ribs; a misshapen skeletal beast with too many limbs and joints that bend backwards, a screaming skull floating, a skeleton in a suit backing away in horror as plants consume another.

\smallcaps{Community vs Party:} Adversarial. I want to chase out the bleak loneliness of death and the grim-darkness of being a skeleton, trapped in a timeless world, being slowly ground down against each other. \sidenote{this was a key decision. A game about a special forces unit in a Necromancer's Army would be very different. And possibly more fun?} 

\smallcaps{Key Theme:} Change. The world and environment are unstable and shifting. \footnote{this was chosen partly by default as non of my other examples in the Theme Deconstruction section seemed to fit.}

\smallcaps{Secondary Theme:} Scarcity. The world is basically a wasteland with only a few key items belonging to the players. This combined with Change suggest dunes blowing. What makes the sand? The ground down bones of ancient skeletons of course!

\smallcaps{Palette:} Emphasis on bones and the limited things the skeletons bring through from the living world.

\smallcaps{Statistics:} Lets go for a simple variant on body, mind, spirit as BONES, MEMORY, GLUE

\smallcaps{Experience:} Let's go for a simple method. At the start of each session, each player gives a brief recap of what they did and marks a new move. No experience is tracked.

\smallcaps{Playbooks:} The key imagery suggests a few already. Two points spread across three stats gives six unique statlines, lets use them as the basis. Let's give each playbook a unique glue to add a bit of imagery to split them up, and lets assign each playbook two of St John Ross's 12 RPG archetypes,\footnote{These are taken from the Risus Companion. I heavily recommend buying it, especially if you are curious about GMing tips for laughter-inducing, surprisingly deep games} it will give the playbooks a bit more depth.

\begin{itemize}
\item BONES 0, MEMORY 2, GLUE 0  
\myitemend Splint+clothes  	--	Persuading			--		Gadgeteer
\item BONES 1, MEMORY 1, GLUE 0  
\myitemend Beast+wires			--Wilderness survival  --		Intrusion
\item BONES 0, MEMORY 1, GLUE 1  
\myitemend Mage+fire				--Communication+Protocol--	Scholarship
\item BONES 2, MEMORY 0, GLUE 0  
\myitemend Knuckles+scrimshaw  --Athletics		--			Drive/Ride
\item BONES 1, MEMORY 0, GLUE 1  
\myitemend Glom+guts			--	Medical		--				Combat
\item BONES 0, MEMORY 0, GLUE 2  
\myitemend Skull+plant			--	Detection	--				Weird
\end{itemize}

\subsection{Basic Moves}

\smallcaps{Struggle}
When you struggle for control of something, roll+Bones:\sidenote{So many decisions here. First is that since there are only three stats I want to limit it to three thematic ways to take action, the trigger here covers combat, but also argument, intimidation and person versus an uncaring universe. Second is that struggling for control is dangerous, so damage is delivered on nearly all options. Third is that, for the basic moves at least, I'll prescribe a 6- result to help the players expectations stay in genre. Fourth is the introduction of a new mechanic, just adding a little extra spice to the dice roll. With only three moves I don't want them to feel too samey in their results.}\\
On a 10+ exchange damage and take control of the thing\\
On a 7-9 exchange damage, take control of the thing and give up a position, something valued or an honest answer to your attacker.\sidenote{you may feel that an `honest answer' is the obvious least-cost choice, but it acts as a good way for the MC to inject loaded questions to draw out characters. The other options are protection for players who hate being put on the spot like that.}
\\ On a 6- the result is as 7-9 but you don't get control.
\\If you also roll a double on any result, don't take damage.

\noindent
\smallcaps{Unleash}
 When you unleash a memory to try and shape something out of the dunes Roll+Memory:\sidenote{I can't have a limbo based game without good and bad memories showing up, and this is a great place for the change theme to be contrasted. Name reduction is a type of HP/manna clock and seemed a thematic way to avoid memory spamming. At this point in the design I'm not sure if I will allow a way to recharge your name with fresh letters, or if the pacing should be to allow one use per session (on average) for a six session campaign.}
\\On a 10+ it is there and will not change, ask the MC how many letters to reduce your name by.
\\On a 7-9 as 10+ but it is weak, temporary or wild.
\\On a 6- something bad is unleashed instead. It will haunt you.

\noindent
\smallcaps{Take Damage} \sidenote{ a little bit of randomisation which means that weapon descriptions can have fixed damage. It is primarily to cover the genre image of a skeleton pulling itself back together from a thousand pieces. The suitable repair is supposed to generate a hook/pressure for the player character to take action.}
 When you take damage, even zero damage after armour, Roll+Glue
\\ On a 10+ it's not so bad, take the damage but reduce it by one or take take the 7-9 option.
\\ On a 7-9 take the damage OR ignore damage but reduce Glue to zero until you suitably repair it (see playbook)
\\ On a 6- take the damage. Your armour fails to do anything.  OR ignore damage but reduce Glue to zero until you suitably repair it (see playbook)

\smallcaps{Decision:} No move for helping/hindering another player. It's a struggle out there. 

So that is three stats that have led to three basic moves (and three little extra mechanisms - rolling doubles, reducing name length, repairing Glue). That is ok for now, but I hope to tighten that up with a bit of playtesting. I have not written a basic move for "get/buy items". I may make it a playbook specific one, or wait and see what types of items the playbooks suggest are needed. 

\newpage
\subsection{Playbook: Splint}
BONES 0, MEMORY 2, GLUE 0  
Name - six letters long
\\ Glue - clothes . Repair by sewing, patching (if you can find the needle and thread) or replacing items of clothing (if you can take them). 		
Your job in life is mixed up in your identity, such that some of your bones are replaced by items symbolic to you. 
\\
\smallcaps{Start Move: Prosthetic}: Start with two implements (choose anything from the equipment list)\footnote{Equipment list: Wrench, till receipt roll; typewriter; chef knife; teapot; stethoscope; pens pens pens; mop; hard hat and boots; camera; measuring tape; cutting shears; every car key; globe; computer mouse; golfclub; bit of pipe; instrument; clock; stethoscope}. Implements are always available for use, but cannot be permanently removed or given to others. Say what body part they replace.
\begin{itemize}
\item  Junkyard: detail another implement that is now part of you
\item  Substitute: you can `consume' a suitable item to heal three damage
\item  Donate (needs Substitute): Spend three health to give anyone a previously substituted item
\item My Uniform: Increase Memory by +1 permanently
\item  Strangely Dishonest: When you lie to their face Roll+Memory. 
\myitem On a 10+ they buy it, and it may even change the world. 
\myitemend On a 7-9 they buy it for now.
\item Presence: When you \smallcaps{unleash} a memory, you can say who, if present, is unable to look away. \sidenote{making clear in the text which words are have important rule triggers that are not on this page.}
\item Demagouge: When you try to lead a volatile crowd Roll+Memory
\myitem On a 10+ if you meet their needs, they'll meet yours
\myitemend On a 7-9 as 10+, but strictly in that order!
\end{itemize}


\newpage
\subsection{Playbook: Beast}
BONES 1, MEMORY 1, GLUE 0  
Name: Three letters long
\\Glue: Thin brass wires. Repair by scavenging the wires from broken or moving dune monsters.
You've been here a long time, if time even passes here. There are many that have entered here before the `humans', and you are cunning enough to have survived.

\smallcaps{Start with one of:} Big Teeth (+3 damage), Huge size (+2 armour), tattered wings (limited flight)
\begin{itemize}
\item  Lamarkian: Take another starting thing. Duplication allowed.
\item  Reputation: Add four more letters to your name
\item  Cunning: When you hide in the dunes Roll+Mem
\myitem On a 10+ you are undetectable until you decide to act
\myitemend On a 7-9 you are invisible to common appearances
\item  Shoving: When you struggle with another beast ritually over territory or similar, on a 6- you won't exchange damage.
%\item  Pounce: When you commit to pin someone down, roll+Bones
%On a 10+ they are trapped
%On a 7-9 you take damage but they are trapped.
%On a doubles, don't take damage. \sidenote{this needs to be added to avoid the move being worse then straightforward struggle. It dosen't add anything to the move. Perhaps the double mechanic should go, or Pounce is simply not different enough from Struggle to justify existence}
\item  Cage Breaker: When you pick a lock with your long wire tounge, Roll+Glue
\myitem On a 10+ it is delicate and quick
\myitemend On a 7-9 it is neither, but you get the lock open
\item  Call of the Wild: When faced with an constructed obstacle you break or bypass Roll+Bone
\myitem On a 10+ you succeed or the MC suggests an alternative good route
\myitemend On a 7-9 you succeed. Enjoy your new situation! (ask the MC)
\end{itemize}


\newpage
\subsection{Playbook: Mage}
BONES 0, MEMORY 1, GLUE 1  
Name: five letters long
\\Glue: coloured fire. Repair by lighting and building a small fire, and bathing in it.
Obsessed or just passionate, you can't see why everyone doesn't agree with you.
\\\smallcaps{Start with Projection}: You scoop and throw a fireball of desire dealing 2 damage.
\begin{itemize}
\item  Griefer: You pull a blinding cloud of ash and embers out of the ground around you.
\item  Furnace: When you hug someone into your rib hollow Roll+Glue:
	\myitem On a 10+ do 10 damage to them.
	\myitemend On a 7-9 as 10+ but you reduce Glue to zero until repaired.
\item  Hatred: When you spit your fire on someone Roll+Glue:
	\myitem On a 10+ it clings and continue to burn and damage them until you speak to them
	\myitemend On a 7-9 as 10+ but strangely, although it burns, it doesn't actually damage them.
\item  Scholar: When you study something carefully Roll+Memory:
	\myitem On a 10+ you spot something no-one else would. Say or ask the GM.
	\myitemend On a  7-9 you have an idea about it. Ask the GM.
\item  Enlighten: When you are stifled by rules Roll+Glue:
	\myitem On a 10+ you correctly guess who to challenge to unmake the system.
	\myitemend On a 7-9 you correctly guess where the power lies.
\item  Flare: When you close your eyes, you can appear as a small fiery image to anyone you have previously damaged. This power may reach over distance and through walls.
\end{itemize}

\newpage
\subsection{Playbook: Knuckles}
BONES 2, MEMORY 0, GLUE 0  
Name = eight letters
\\Fire: Scrimshaw. Repair by having an artist add scrimshaw to replacement clean bones.
The simplest approach is often the most direct.
\\\smallcaps{Start with Brawler}: Even unarmed you still do 3 damage
\begin{itemize}
\item  Bullish: When you hit someone hard you can choose to knock them backwards a long distance.
\item  Naturally Spiky: Reduce damage taken by 1. This does not stack with armour, use one or the other. 
\item  Athlete: When you push to a physical feat that others couldn't do, Roll+Bones:
\myitem On 10+ you do it.
\myitemend On 7-9 you do it but take 2 damage.
\item  Artistic: As long as you have a working hand and a suitable tool, you can record your feats in scrimshaw (counts for repair glue). Otherwise you must find an artist.
\item  Domination: When you wrestle a large dune monster into submission Roll+Bones: \sidenote{struggle theme again -  no easy loyal animal companions here!}
\myitem On a 10+ it accepts you. Treat it well and it will let you ride it.
\myitemend On a 7-9 as 10+ but it dislikes you and will use every opportunity to escape.
\item  Martial Dancer: When you prepare to fight Roll+Mem
\myitem On 10+ hold 3,
\myitem On 7-9 hold 1,
\myitemend During the fight, spend hold one for one to - move unstoppably; break a grip; block an attack\end{itemize}


\newpage
\subsection{Playbook: Glom}
BONES 1, MEMORY 0, GLUE 1  
Name: six letters
\\ Glue: Guts. Repair by consuming fresh meat or funeral foods.
You always blended in well, and bones are bones are bones right?

\smallcaps{Start with Amalgamate}: When you gut wrap sufficient new bones into any skeleton Roll+ Glue: 
\begin{itemize}
\myitem On a 10+ heal 3 or heal someone else 3 
\myitemend On a 7-9 as 10+ but reduce glue to zero until repaired
\item  Gall Stones: You develop a pot belly of stuffed with bones and sand to digest dry old bones. Consuming a full skeleton is enough for you to repair your Glue.
\item  Blemeye: Your skull moves down inside your ribcage. Increase you base armour by 2. More armour stacks with this.
\item  Cage and Club: One arm ends in a ribcage shield, the other in another femur. You always count as armed with a Large Club, and the shield increases your base armour by 1. 
\item  Gut Lasso: Needs Gall Stones or Cage and Club. When you throw (or spit) a lasso made of your guts, roll+Glue.
\myitem On a 10+ you get them. You can unbalance, disarm them or draw them closer.
\myitemend On a 7-9 you get them but it's awkward. Choose as 10+ but then they choose an option to do to you.
\item  Double Jointed: Needs Cage and Club or Blemeye: You gain a second pair of arms. If this means you have four hands this allows you to duel wield Great weapons. Otherwise it just means you have hands free to carry things.
\item Mother's belly: Needs Gallstones or Blemeye: You have steadily grown larger over time. When you take a destroyed or damage skeleton inside yourself, they can ignore all damage or attacks focused on them. You may reanimate and heal them, using Amalgamate. They will be tied to you by a string of guts until they find and repair their own glue. 
\end{itemize}

\newpage
\subsection{Playbook: Skull}
BONES 0, MEMORY 0, GLUE 2  
Name: eight letters 
\\ Glue: Plant. Repair by filling skull with water and staying in the sun for a short time.
In a world of white sands, what is left of you brings the true death. 

\smallcaps{Start with: Manifest:} When you need to be more then a slowly floating skull Roll+Glue
\\On a 10+ you manifest a knotted viney body, as strong and fast as any skeleton.
\\On a 7-9 Choose one: you are small, limbless or slow
\begin{itemize}
\item  Rooted: When you appear to remain motionless, you may push a limb out of the dust three body lengths away.
\item  Infectious growth. When you deal damage and roll doubles you can choose to ignore their damage or to infect them. Infected take one damage each turn ignoring armour.
\item  Tendrils: No-one can sneak up of you. Your attacks can entangle.
\item  Blossom: When you exert power through an infected, they blossom with large flowers. They become the focus of the attention of anyone around.
\item  Detection: At start of session, When you do specialty roll for hold, spend hold for stat bonuses
\item  Weird: When yo do specialty it counts as base move using stat. The bonus is power of springboard fiction. prompt is cast of NPCs.
\end{itemize}

More work to be done here I'm afraid!

\newpage
\subsection{Weapons, grave Goods, Beasts, Gangs, Unleashed Memories}
