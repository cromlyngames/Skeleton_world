\documentclass{tufte-handout}

\title{An example of world and game making for Powered By the Apocalypse\thanks{Inspired by the amazing SimpleWorld of Avery Alder}}

\author[Cromlyn Games]{Cromlyn Games}

%\date{28 March 2010} % without \date command, current date is supplied

%\geometry{showframe} % display margins for debugging page layout

\usepackage{graphicx} % allow embedded images
  \setkeys{Gin}{width=\linewidth,totalheight=\textheight,keepaspectratio}
  \graphicspath{{graphics/}} % set of paths to search for images
\usepackage{amsmath}  % extended mathematics
\usepackage{booktabs} % book-quality tables
\usepackage{units}    % non-stacked fractions and better unit spacing
\usepackage{multicol} % multiple column layout facilities
\usepackage{lipsum}   % filler text
\usepackage{fancyvrb} % extended verbatim environments
  \fvset{fontsize=\normalsize}% default font size for fancy-verbatim environments

%for fancy lists
\usepackage{tikz}
\usetikzlibrary{shadows}
\newcommand{\mylist}{\tikz[overlay]\draw(-.2,-.2)--(-.2,.5) [path fading=east](-.2,.15)--(.1,.15);} %adds the |- shape to the start of each list item
\newcommand{\mylistend}{\tikz[overlay]\draw(-.2,.15)--(-.2,.5) [path fading=east](-.2,.15)--(.1,.15);} %adds the |- shape to the start of each list item
\newcommand{\myitem}{\item[\mylist]} %defines the scope of the mylist command to be 2nd level sublists
\newcommand{\myitemend}{\item[\mylistend]} %defines the scope of the mylist command to be 2nd level sublists

% Typesets the font size, leading, and measure in the form of 10/12x26 pc.
\newcommand{\measure}[3]{#1/#2$\times$\unit[#3]{pc}}

% Macros for typesetting the documentation

% Generates the index
\usepackage{imakeidx}
%\makeindex[name=moves, title={Index of moves}] % on reflection, not needed. 
\makeindex[name=stuff, title ={Index of elements, items}]
\makeindex % general index for playbooks and stuff



% Standardize command font styles and environments
\newcommand{\doccmd}[1]{\texttt{\textbackslash#1}}% command name -- adds backslash automatically
\newcommand{\docopt}[1]{\ensuremath{\langle}\textrm{\textit{#1}}\ensuremath{\rangle}}% optional command argument
\newcommand{\docarg}[1]{\textrm{\textit{#1}}}% (required) command argument
\newcommand{\docenv}[1]{\textsf{#1}}% environment name
\newcommand{\docpkg}[1]{\texttt{#1}}% package name
\newcommand{\doccls}[1]{\texttt{#1}}% document class name
\newcommand{\docclsopt}[1]{\texttt{#1}}% document class option name
\newenvironment{docspec}{\begin{quote}\noindent}{\end{quote}}% command specification environment

\begin{document}

\maketitle% this prints the handout title, author, and date

\begin{abstract}
\noindent
This document describes some ways to make  a Powered By the Apocalypse game.
It should support reskins, Game-masters's\footnote{I will use GM throughout this document for abbreviated clarity. I tend to use MC in hacks I'm writing.} looking for setting ideas, and help building a custom class.
It also supports deep hacks that play with the fundamentals of the game.
If you were hoping for Skeleton World, the PBTA rpg, it's at the end.
\end{abstract}

%\printclassoptions
There's three ways to use this document. One is around a table and build the game the table wants to play together. Another is to pick and choose bits from it to help you explore an idea. I like to pick randomly at each stage, forcing just the right amount of constraints that there can only be one solution for each thing from stat to archetype to individual move, taking me outside where I can get to normally.

\section{Size}
Choose one:\\
Tight game: one themed additional mechanic\footnote{An example is Apocalypse World, which ties everything to destruction} \\
Balanced tension game: two themed additional mechanics\footnote{An example is Urban Shadows, which introduces Debts and Corruption}\\
Focused big game: one themed additional mechanic, second theme present as a stat.\footnote{An example is the Sprawl, which adds synth as a stat for cyberwear}\\


You need a fast, nuanced fail / weak hit / strong hit generator\footnote{The standard method is two six- sided dice with margins set at -6, 7-9 and 10+. This can be easily mapped to other types of dice. Cards are slower but can work. Dream Askew used tokens. I'm intrigued by rock/paper/scissors but there are game flow implications}
 and a theme and a genre. This will give you an Agenda and Principles. These will give you GM moves. 

Choose at least two and normally four or five:
Stats, Basic Moves, Playbooks, Development, Scenario Moves, Inserts, Tags.

\section{Genre}
What genre (or mashup) do you all want to play in?
You may know, if not pick one or two from each below:
\begin{multicols}{3}
ancient legends\\
high fantasy\\
low fantasy\\
grim fantasy\\
age of sail\\
mediaeval\\
pike and musket\\
steampunk\\
Victoriana\\
colonial\\
1920s pulp\\
superhero comic\\
weird world war\\
biopunk\\
cold war\\
diesel punk\\
urban fantasy\\
slice-of-life\\
alternate history\\
dystopia\\
hard sci fi\\
solarpunk\\
cyberpunk\\
post-apocalypse\\
space opera\\
\end{multicols}

\subsection{(Optional) Party is}
\begin{multicols}{3}
circus\\
first responders\\
mercenaries\\
city guards\\
criminal gang\\
merchant convoy\\
monster hunters\\
smugglers\\
traders\\
investigators\\
boarding school\\
university dept\\
neighbours\\
village\\
operatives\\
rebels\\
police\\
refugees\\
outsiders\\
bureaucrats\\
\end{multicols}

\section{Theme}
What feeling or thing is this game about?
Choose one or two, eg
"What would you do for X"
or
"Is X more important then Y?"

\begin{multicols}{3}
balance\\
belief\\
change\\
comfort\\
destruction \\
corruption\\
family \\
freedom\\
friendship\\
glory\\
honour\\
hunger\\
justice\\
knowledge \\
law/tradition\\
love\\
pain\\
revenge\\
unity\\
survival\\
violence\\
wealth\\
.......\\
.......\\
\end{multicols}


If you aren't sure, take a read of the Theme Deconstruction Section at the end of the document. Every theme comes with a additional GM Principal, GM move and a possible mechanic to stir into the game. I've also introduced three other novel little mechanics in the basic moves section of the Skeleton World rpg. The limit is your play tester's patience.

\section{Stats}
Aim for the fewest number you need. Keep them simple, memorable, and immediately highlighting the difference in how different characters approach challenges. A good example is Versed, Young, Gendered, Wyrd from Sagas of the Icelanders. It clearly spells out what is the important differences between two characters.

Choose one set and rename for your genre, or mix and match between sets:\\
1. Fighter, Thief, Wizard\\
2 .Social Circles, Danger Triangles, True Squares, Stitched Crosses \\
3. Assertive, Persuasive, Curious, Methodical, Confident\footnote{this one is loosely based on the Big Five Personality model}\\

Or choose them by Genre, either on skills, common environments or to support basic skills you expect all players to need:\\ 
a. Space, Robots, Lasers,  Feelings\\
b. Honour, Rice, Ki\\
c. Fate, Power, Control \\

\subsection{Allocating stats}

The only essential here is to prevent players breaking their character when they are still unfamiliar with the game. Some games allow free allocation of stats (point buy method), some have the stats built up from a personality questionnaire.

The easiest, reasonably robust, method seems to be the Apocalypse World of asking a player to choose between three or four different stat-lines. This allows you to avoid one stat being boringly strong to start with (these games run on partial failures). It also allows nuances, since the different stat lines offered don't have to sum to the same value. In some settings, capability in something may be more common. In game design terms, for the fictional arc you want to send player characters on, the opposite may apply. In the Star Wars setting, force-sensitivity is rare. In a Star Wars game (or film), force-sensitivity is common for important characters to have or develop, while database hacking remains a rare and valuable skill. 


\subsection{If no stats}
If you are not having stats, there's three ways to  handle it\\
1. Just roll and rely on the nature of the moves for variety. In this case you may want to reduce the dice roll targets by one (fail on a -5 not a -6 ect)\\
2. 'Roll for advantage'. Roll three dice and discard the lowest. This is moving the game close to a dice-pool mechanic as used by Blades in the Dark.\\
3. Mark each move for basic/good/advanced. Basically treat each move as it's own stat, similar to a skill based game engine. Examples are Exo-adventurers or World of Dungeons: Turbo Breakers\\
4. Use ephemeral stats like 'Roll+bodies searched' or 'Roll+Hands free'. Be careful with this. If that information is going to be steady from scene to scene (like number of eyes) just make it a stat and let them write it on their character sheet.

\section{Basic Moves}
Ideally start players with seven or less moves to remember. I tend to split it as five basic moves, two playbook moves. I tend to have one basic move per stat. The core of a basic move should be "Take thematic action"\footnote{EG Apocalypse World has you "open your brain" to the maelstrom to shift the status quo, while Urban Shadows has you "let it out", risking probable corruption in return for supernatural power}. Examples of thematic basic moves are offered in the Theme Deconstruction section\ref{themedecon}.
All Basic moves should be thematic. They should also support play. Often a player character will want a genre (not generic) version of:\\
a. Get information\\
b. Persuade/seduce\\
c. Help/Interfere with another player\\
d. Buy/get things\\
e. Mortal peril \footnote{well, they might not want it, but they may need it}\\

\subsection{Get information}
This is often in the format of roll+stat and get hold, spend hold to ask questions. The use of hold allows players to develop lines of inquiry as more information becomes available in the scene, rather then the distraction of forcing rerolling many times. 
The questions available to be asked are often listed and the choice of what can be asked is important, it tells the player what their character thinks is important about the genre. The Sprawl (mission based cyber-punk) has "Read A Situation"; Urban Shadows (political urban fantasy) has "Figure Someone Out".
Players will act on the information they get. If you make them ask thematic questions they often end up taking thematic action.

\subsection{Persuade/Seduce}
The challenge with this move is  balancing the requirements of the genre/theme (in a Noir game\footnote{there is a Noir World hack but I have not played it. The examples I draw from the genre are my own. Don't let me put you off someone else's work!}, the Femme Fatale must be able to manipulate the other archetypes) and , zooming out a little, balance the social contract of the table and the expectation that players have of their own agency. Storyteller-type players might be happy to sacrifice character agency in return for the ability to add plot twists or elements of world building. Actor-type players might be happy to give up all power outside their character's internal thoughts and decisions, but will hate anything that intrudes on the territory that they do control.\footnote{It is good to be clear about your approach in your game introduction. The type of player a person is or wants to try being can vary by game, table and the phase of the moon.}

There are a few options:\\
1) Make the move apply to NPCs only. Get players to negotiate if their characters act on each other. This is only going to be ok in games where the player characters are supposed to be aligned teamworkers and where persuasion isn't a big thing in genre, like Fellowship\footnote{Citation needed. The actual text of the Talk Sense move in Fellowship does not specify NPCs only, but it covers things that Players normally agree between themselves anyway.}. \\
2) Make two moves, one for manipulating NPCs and one for manipulating PCs. This is typically set as a standard move for NPCs and offers an extra carrot/stick for PC players to go along with your plan. This can work well for cinematic games where the player may want to get their character into thematic trouble.The Regiment has a move called "Impose Will" which uses only stick, but who plays an Army themed game and doesn't expect to need to take orders?\\
3) Make one move, that controls NPCs and other PCs. Ignore PC agency. This will earn you some hate-mail, but can work well in games where players have other ways to influence the story then just their player actions. 

\subsection{Help/Interfere with another player}
The basic form is "When you help someone or hinder them, roll your relationship stat with that person. On a 10+ add or subtract 2 from their roll. On a 7-9, the MC will name a cost, if you accept this cost, then add or subtract two from their roll.

The exact nature of the relationship stat shifts between games, normally plugging into other mechanics. In Apocalypse World the Hx stat is constantly changing and is a source of character advancement. In the Sprawl the Links stat is mostly static, but sets up the initial threat clocks of different mega-corps. In Urban Shadows it is faction based, but also driven by the powerful Debt moves.

In a PBTA game, I think it is best to avoid allowing players to 'defend' against each other by making opposing rolls. It makes the format of moves break down and some complex snarls in the fiction to develop which distract from moving the story forwards. It's like the tennis ball hitting the net instead of being passed back and forth in a volley. Instead, the help/interfere move of the second player alters the chance of the first player changing the fiction the way they want to.

\subsection{Buy/get things}
I have included it here, because most PBTA rulebooks do address it, but as story driven themed games, shopping and equipment and gear tends to be of secondary importance. Examples of thematically addressing it are The Sprawl's move for getting additional cyberware installed where you have to choose between dangerous back alley docs or villainous megacorps. 
Apocalypse World has a barter move to cover sourcing 'some particular thing where it's not obvious you should be able to go buy it just like that", as a way to convey the Scarcity theme of the world, but allowing the players to perhaps drive the story further than a blanket 'no, you can't buy that.'. 
Both moves are all about giving players what they want, perhaps at a cost, and giving the GM some more hooks or levers to pull to put the characters into more interesting situations.

\subsection{Mortal Peril}
There are two branches to this move. The first is a catch-all "Take action under pressure" move that can be the main point of weakness in a game if the GM is forced to rely on it too much. 
The second branch is the move that kicks in should a player character actually die. It tends not to be something that comes up often in pbta games but when it does it should be important to the story, and the rules often support a gm making a big deal of it. Urban Shadows gives each playbook it's own 'in case of death' move. Dungeon World gives the player the option to become 'Death's Own', accepting that that character WILL die, but at the climax of a particular narrative arc.
The common basic move for 'running out of health' allows you to choose between taking a new character, changing playbook or taking a permanent hit to your character (normally a stat reduction, hopefully with a fictional justification like losing a leg). 

\section{Agenda}
The GM Agenda is always:

\begin{itemize}
\item Play to find out what happens
\item Don't waste your players time
\item Deliver the world according to the genre setting
\item Deliver action in line with the theme
\end{itemize}

In other words, don't railroad your players but do respect the logic of the story world, the fiction. Don't pre-write a story that ignores them affecting the world and don't space out or make them grind for the cool moments `just because it feels like they haven't worked hard enough.' It can be a weird temptation as a GM, perhaps because the creativity required can feel tiring and you just want them to roll grinding dice checks for 15 min while you take a break. We'll cover techniques to help avoid that tiring feeling in the GM principles section.

\section{GM Principles}
\begin{itemize}
\myitem sprinkle evocative details everywhere 
\myitem make the world seem real
\myitem name everyone, make everyone human
\myitemend build a bigger world through play
\myitem create interesting dilemmas not interesting plots
\myitem address yourself to the characters not players
\myitem be a fan of the players characters
\myitemend destroy your creations, don't protect them.
\myitem make your move but misdirect
\myitem make your move, but never speak its name
\myitem ask provocative questions and build on the answers 
\myitem sometimes, reflect a question back upon the players 
\myitemend think off screen too
\myitemend + Additional from the theme (see Theme Deconstruction section)
\end{itemize}
Rewrite these to suit your genre. An example is "Barf Forth Apocalyptica" replacing "sprinkle evocative details everywhere." The word choice isn't critical, but you want to tone of the writing to help get the GM in the mood.\footnote{"getting you in the mood" is why I think Apocalypse  World calls the Games Master the Master Of Ceremonies. You are not about controlling the situation. You are about introducing each player character as an act that gets their own time in the spotlight and keeping the show moving.}

It may seem weak that the GM principles are so similar between different genre games and experiences, but that's becuase the first four are about world building, the next four are about managing your ego (MC, not GM!) and the last five are about ways to improvise that aren't as tiring as thinking up a plausible monster biology every five minutes. The final one of course, will vary with theme.

I am not going to go into a deep explanation of them all here, as I am assuming you've played a few PBTA games before trying to write one. If you really need help, buy and read Apocalypse World as it has a strong guide with many examples of getting it right, or how to correct if you feel you have got it wrong.

\section{GM moves}
\begin{itemize}
\myitem put the spotlight on someone
\myitem seperate them
\myitem put them together 
\myitemend make their lives complicated now
\myitem give them a difficult decision to make
\myitem offer an opportunity, with or without a cost
\myitemend offer stuff that's painfully expensive but good
\myitem tell them the possible consequences and ask
\myitem turn a failed move back on them
\myitem use up their resources 
\myitemend activate stuff's bad side
\myitem use a front or threat move
\myitemend + Additional from the theme (see Theme Deconstruction section)
\end{itemize}

In the same way, I am not going into detail the GM moves or how to use the standard ones. The first four moves are about ways to move player characters about. There's no reason to have them all in the same place or working in perfect agreement all the time. Think of the beautiful jump-cuts of a heist movie where a separated team work together versus the intimate squabbling of a sitcom or the rolling partial scenes of a Shakespearean Play. It's something that many players have been trained out of or picked up from cultural osmosis (never split the party!), and the onus is on you to shuffle them around to keep their characters uncomfortable and the players grinning.
The next four moves are all about giving your player's agency, opportunities to make meaningful decisions. My favourite one I ever managed was a cyberpunk game where the player was hanging off a crashed skyrail. ``You see someone tumbling towards you, that punk with a hook for a hand. You can try to grab him as he goes past , but you can't quite be sure which of his flailing hands you'll get. Otherwise he's a deadman. What do you do?'

This has obvious overlaps too with the next four items: setting up clear consequences. Generally, players won't mind horrific consequences if they knew the risk going in. ``you fail to close the portal, in fact it rips wide open, it's now far too wide for just the three of you to stop anyone coming out. What do you do?'' In this vein, reminding them that stuff has a bad side makes it seem less arbitrary when you activate it later. ``If you use grenades here, you risk destabilising the whole reactor.'' This also blurs back into ``Tell the consequences and ask''.  The level of overlap here is your friend, don't get too worried about the specific move. Just stab your finger on the list somewhere and do that.

The last two items are going to be more specific to your story, genre, setting and theme. They are in fact your primary ways as a GM (rather then game designer) to bring the theme out.  Even if the enemy is as simple as a Giant Rat, the threat move it comes with should be different in a cyberpunk game and a low-fantasy one. What should each be?

\section{Playbooks}
I normally aim for 6 to 12 playbooks, or do without them altogether. If you do that, then you can either let  players pick from a common list of moves (Class Warfare for Dungeon World), have only shared basic moves (World of Dungeons), or use only custom moves awarded in response to player actions in that session or the previous session (Experimental, but could work?).

Generally speaking I advise using playbooks. As a designer they make it easier to balance characters and helps keep the game more genre aligned and thus produce stories that feel like the table is working together to tell it. For the player, a playbook lessens the amount of information to process before making those first choices and gives you an easy to understand initial role in the story. Confident players make moves.
Why six to twelve? You want to give the last to choose player some choice, especially if they (or you) don't like two or more of the same playbook. I suggest you aim for everyone to have different playbooks as it feels more `special' for the player to have something no-one else can do. That connection is the start of the dice rolls meaning stakes.
If you are making playbooks then there's a bunch of ways you can space them out.  The first is by writing down archetypes of the genre (so in Film Noir we have the Detective, the Mobster, the Suit, the Femme, the Rival Cop, and then, playing on the genre time period and theme of guilt, maybe the Priest, the War Hero, the Waif, the Outsider). 

Next I'll start sketching out the character creation bits before getting too hung up on moves. 
As a player, I like the choose an option list approach in playbooks, as they tend to keep me on theme and provide seeds to get started quickly. My current recipe for these is five finished options, three rule based options and one blank to allow those who really want to to fill it in themselves (giving them a sense of ownership). \footnote{For example. Choose a Name: Fizzok, Hardcrust, Bluebell, Edbanger, Sappo, a flowery name, a punky name, a stolen name, \_\_\_\_\_\_\_}  The type of things chosen don't have to match between playbooks. Generally speaking I recommend following Vincent and Megeuy Baker's approach in Apocalypse World and basing phraseology around `appearance' and not `actuality'. 

Give people some starting equipment, and questions to ask other players. It helps tell those other players a lot about this one, and the answers can be used to get things moving from the start.

\subsection{modular playbooks}
The Dungeon World online content ecosystem approaches modularity with Compendium Classes. These are stripped back, single focus playbook addons that offer a couple of moves and/or a unique mechanic. This makes them ideal to cover situations like what happens after being bitten by a weregator or the assembled effects of an artefact suit of armour. The Dungeon World add-on book, Class Warfare, allows you to build a playbook from multiple compendium classes.

One example of a game that uses modular playbooks from the start is the Urban Modern Fantasy hack of Dungeon World. There you build a character from a stat focused module and a urban fantasy focused module (eg a Genius+Ghost or an Athlete+Medusa).  The Happiest Apocalypse in the World takes this even further. In this game of razor blades in candyfloss and themeparks with madness bubbling beneath, you build your character by stitching together three modules: personality type + professional background+guest/staff.

Modularity suggests the potential for adding Traveller style life-paths to character creation, but I've not seen that done yet. Modularity presents a challenge for the production of useful character sheets or print-and-fold playbooks. From a table-top perspective, you do not want to slow the game down by adding a minute of page flicking to every interaction, it really pulls people out of the game. Try to at least consider balance between the combinations.\footnote{I have seen one GM complaining that one of their players had used Class Warfare to build a heavily buffed character with only one fiction changing move (beyond the basic moves). For me, that's not a problem, but I can see how it might constrain the levers a GM has to play with to deliver a satisfying experience}. An option that is completely outclassed should be removed or rewritten. 

\subsection{sketching playbooks mechanically}
When I'm making a new hack, I like to do the following as a first sketch. If I am using stats I'll write down a list of primary and secondary stats (eg Aa Ab Ac, Ba, Bb, Bc, Ca Cb Cc) and try to fix archetypes to it and fill the gaps. If you treat them as primary and secondary, then for N stats you get N squared slots (three stats is nine playbooks, four stats is sixteen). If you drop doubles that takes you to six and twelve respectively. If you keep doubles but instead treat Ab and Ba as the same then you get six and ten playbooks respectively (4+3+2+1). It's satisfying lonely fun. 

If I have other mechanics in the game I might do the first sketch list with that. Cyber\_peripherals, for example, is about Gangs, Holdings, Favours and Cybermods, and each of those things is important enough to be in the core rules. I wasn't going to have sixteen playbooks. So the list started Gangs-gangs, Gangs-Holdings (the goon), Gangs-favours (the charity worker)ect, but I didn't fill in all the slots, just enough to represent everything in most four player games.

Then I'll will assign the playbook a GM prompt from the list below. These are things the playbook brings to the table that provide, prompts, hooks, buttons or leashes that help the GM keep the story moving.  They are a good thing to ask provocative questions about.

\begin{itemize}
\item start of session momemtum triggers eg resource shortages
\item Cast of NPCs  eg regular customers
\item catalyser of conflict eg Gunlugger
\item expand map eg Driver
\item Instigate threat eg Vampire Hunter
\item fixed point to defend/orbit eg Hardholder, king \footnote{this goes against classic dungeon crawling where the party must stay together and so anything like this is an impediment to the story.}
\item need for supplies eg Savvyhead
\item supports risky play eg healing
\item countdown urgency eg Vampire
\item broadcast npc motivations eg observant\footnote{broadcasting motivations even to a single player normally means all other players hear it, and it helps bring the world to life a little more for everyone}
\item Mostly obedient entity eg Beast-master.
\end{itemize}

Why go to this effort? It makes writing moves easier. If I know this playbook is about the Detective archetype, that the playbook's primary stat is Street and secondary is Smarts and it engages with the Clue mechanic but not the Guilt or Resources mechanics, and brings the Start of Session prompt for the GM, then I've got a good lead. You can do the same spacing out exercise inside the playbook too. 

Lets say for the inital draft the Detective starts with one Street move and one start of session Clue move, and has options to learn another two Street moves (three total for primary), two Smarts moves and one more Clue move. That's seven moves which is a good set. \footnote{ Often more then Apocalyspe World playbooks get, although they have complex unique playbook mechanics. Dungeon World by contrast, will often hit fifteen or more simpler moves since frequent levelling up is a core part of the Hero's Journey theme.}

What are three characters that this playbook should be able to build? The crumpled, world weary but insightful genius, the hardboiled ex cop simmering with anger, the flawed but charming (and patient) interrogator. The crow, the bull and the spider.

Lets give the Crow archetype the second Clue move (something to do with last minute insight?) and give the Bull and the Spider aspects one move each for Street and Smarts. Of course a player might mix those up and decide to focus on all the Smarts moves, or even, if you allow the option, learn a move from the Femme book and specialise in uncovering details in relationships. Building in a way to allow a move to be learnt from a different playbook goes a long way to allowing players to explore the exact character they are playing. It works as long as the playbooks are spaced enough to prevent `obvious' choices.\footnote{the Happiest Apocalypse on Earth allows a new move from your playbook for 5xp, and a new move from a different playbook for 10xp. This seems an elegant way to do it. It guides the player in picking up their core playbook moves first, but allows a more developed character a little flexibility}
 
The trick with this of approach is to constrain it just enough that as designer you can only see one obvious answer, not to give you a headache trying to fit some arbitrary, impossibly difficult combination together. 
When writing some hacks, I might generate each move for the play book randomly, trusting to statistical clumpiness to give each playbook an emergent theme. I'll give examples of that in the next section.
 
 \subsection{Anatomy of a move}
If you write a custom move, roll+WIS
\\On a 10+ it is thematic, useful and short 
\\On a 7-9  it is two of the above but not the third.
\\On a 6- it is a failure. Rely on the GM to try and salvage it.

Sometimes I will have an idea for a particular move for a playbook that I will make sure is there for the first draft. I also like the depth you can get from a wider range of slightly conflicting moves. Randomly rolling them can work for me.

Choose one from each list:

\subsection{Trigger and move shape.}
\begin{itemize}
\item when you do something related to (specialty) roll for bonus
\item when you do (specialty) roll for/get hold. spend hold for bonus
\item you have the ability to (active power). it counts as base move using (stat)
\item you have (passive power with constant effect)\footnote{by `power' I mean something that changes the fiction - cast light or walk through walls for example}
\item you have a (thing). when applicable it adds two bonuses
\item Straight bonus
\item When you interact with another player, bonus
\end{itemize}

\subsection{Bonus is}
\begin{itemize}
\item bonus of +1 to stat  \footnote{some people disapprove of straight +1 to stat bonus moves, but they are useful for a player who doesn't want to manage more rule complexity yet}
\item bonus of reroll a fail
\item power of springboard fiction (eg cast light)
\item power of mechanical fiction (eg healing, clue mechanic)
\item get knowledge 
\item juicy list
\item juicy complications \footnote{normally choose two of three that don't happen.}
\item oddball juicy list \footnote{the favoured move shape of Urban Shadows. Choose three options out of four, where one of the four options is a an oddball.}
\end{itemize}

\subsection{Prompts}
These may look familiar from before. They can work well for individual moves to ensure the story moves forward. Prompts may be part of the 7-9 complications, -6 consequences or even +10 bonuses!
\begin{itemize}
\item start of session momentum triggers 
\item cast of NPCs 
\item catalyser of conflict 
\item expand map
\item Instigate threat
\item fixed point to defend/orbit
\item need for supplies
\item supports risky play
\item countdown urgency
\item broadcast npc motivations
\item mostly obedient entity
\end{itemize}

Let's look at that Detective starting Street Move. I roll on the lists above randomly:
When you do something related to (specialty) bonus. Bonus is juicy list. Prompt is Countdown Urgency. I interpret that as:

When you take on a new case, Roll+Street
\\On a 10+ choose two, \\On a 7-9 choose two, but one will be lost soon.
\begin{itemize}
\myitem you know you have an informant good for this
\myitem you get a cash down payment up front
\myitemend you see a link to another case
\end{itemize}

Lets roll again for the Clue starting move\footnote{my initial thought before rolling was ``If there is a clue token at stake in this scene, you have a hunch it is available."\\ It is short and engages the theme, but on reflection it feels like it'd result in a frustrated player as much as a frustrated detective character.''}

You have a (thing). when applicable it adds two bonuses. Bonus of reroll a fail, Prompt of catalyser of conflict . Extra bonus of get knowledge.

You have a stained, run-down office you can barely afford. Choose one thing it has: Fire-escape to alley; loyal secretary called Dolores, filing cabinet with years of neat notes and newspaper clippings.
When you spend the night in your office, you can ask anyone one question for each Clue token you hold, as though you are back there at the scene. When you act aggressively on the answers, you can reroll the first failed roll. \footnote{it's certainly thematic, might need editing to be a little shorter and tighter, and will need play testing to ensure it's not a spotlight hog}

In the example above I decided which moves had a roll and which didn't. In my experience so far, about one in  ten of the moves won't survive initial playtesting and will be replaced by something else entirely. This is an initial draft to get you to the table as fast as possible. Again, while I like the rolling method to get me beyond the obvious and avoid playbooks with 14 moves keyed to tiny moments of violence; it's a device to help, not block you completely. If a combination feels truly unworkable, or Lady Luck has given you nothing but information gathering moves, adjust them! The random generators can be pushed further too - write down the dozen or so player character archetypes, write down four specialties and two `things' of each archetype, assign each playbook a mix of two archetypes and roll up a specialty or a thing if the move uses it. 

\section{ Development}
Development is the mechanic and rules system that governs how players can develop their characters.\footnote{If you are writing a one shot game, you probably don't need this. } The general design principle I advocate for this is that most games don't last more then a few sessions, levels, so make sure the characters can do cool stuff from the start. 
It is a mechanic that changes often between different PBTA hacks, indicating that no-one has found a solution that everyone else is happy with. It's a perennial favourite topic for discussion, perhaps because it  ties so closely into the hopes, desires and sense of entitlement of players. 

\subsection{Simple models}
The simplest model of all is no mechanical development. In that case players start with a `complete' character, and any development occurs in the fiction (acquiring contacts, loot, relationships, exotic pets ect). An example (outside PBTA) is Traveler - where the genre setting of you all playing crusty middle aged `professionals' with a ship sized mortgage to feed means that static character skills fit the genre. Using this method  focuses player development urges into the fiction and reduces mechanical complexity. It reduces how complex a playbook can be be (since you don't want to overwhelm someone with a dozen things to memorise right at the start) and dosen't give a player an escape route to develop the playbook more into their style. 

The next simplest is session levelling, where you get an upgrade at the end of every session or every two sessions. This can feel arbitrary and unsatisfying because you don't have to `work for it'. If the upgrade reflects the last session then this is a bit better. It is easy to implement, and to be honest most games try and calibrate for this speed of development anyway. 

Milestone levelling is another option, where you `level up' after hitting some longer term arc/project. The Savvyhead in Apocalypse World kind of does this by default - as they complete projects, the MC will often give them a custom move that ties into that gear. The limits with this method is that is suits a team game with a shared goal much more then players at cross purposes. You really don't want to put the gm in a position where they have to make calls that will let one player milestone up and block another. That kind of bleed does not feel like a good thing at the end of the session.  I think Nightwitches uses Milestone levelling as it suits the mission based, team based nature of the setting.

\subsection{XP models}
Most PBTA games use XP points, with normally a few different things feeding into your XP total. I am fairly neutral on these. Pick whatever drives thematic behaviour.

Dungeon World and Monster of the Week gives you XP on a `failed' roll.
They both give you `top up' xp if you can successfully answer three thematic questions at the end of session.\footnote{ie did you find treasure, explore the world, learn something new ect.}
The Happiest Apocalypse of Earth allows the MC to offer players XP in return for increasing the madness of their situation.
The Sprawl plays against the team dynamic, and offers XP when you follow personal directives that put the group at risk.
Apocalypse Word gives you XP when you max out and reset your relationship stat with someone. Dungeon World does something similar with bonds. 
World of Adventure has adopted Keys - which give small XP when triggered and big XP when completed and replaced.\footnote{Ie Key of the Merchant: \\ Gain 1Xp when you negotiate a trade that makes profit by the end of session. \\ Buyoff: 3xp and change key When you refuse to part with something at any price.}
Apocalypse World also gives you XP when you roll with a highlighted stat. The twist is that other players choose the highlight as a hint what they'd like to see at the table. 

\section{Scenario Moves}
This is a way of bulking out the basic moves to keep them from getting boring, without asking your players to memorise a ton of moves right at the very start of the game.
The Sprawl uses these to cover less commonly occurring, but thematically very important situations in the cyberpunk stories - getting the job, dying and getting cybermods. Sometimes in that order.
Apocalypse World uses them to cover complicated motorcade warfare without relying on a single basic move all the time. 
Dungeon World could perhaps benefit from a `Scouting mission' scenario move set covering falling, sneaking, climbing, being on fire ect. Stanberg has produced a nice set called "Drowning and Falling". 


\section{Inserts}
These are a different way of handling complex `modules' of rules that are not moves. Dungeon World uses them for pet familiars. Apocalypse World uses them for gangs and vehicles. They are literally a self contained page of rules that you can append to a playbook to handle something special. 
You know the Driver starts with a special vehicle, but it is easy to see how many player characters may acquire one during play and need to insert the rules for their whiskey powered motorbike into their play book.

If you want to free a specific thing from a playbook archetype, or in playtesting people keep insisting on owning a particular thing/gang/pet -  an insert may be the way forward! 
Like a playbook, I suggest a mix of fixed options and thematic rules for name, appearance, personality traits or predictable behaviour of the insert when the player rolls a fail.\footnote{Apocalypse World requires you to assign a custom vehicle a personality. This is not to suggest that the vehicle is alive, but in situations where the dice take over, it does help the GM be fast and consistent.}  A few explicit ways for the player to upgrade the insert are also good. 

\section{Tags}
How do you separate a sharp knife, and a *sharp* knife? Most, if not all, PBTA games use tags. Most of the time they are a `this special rule applies' reminder. An example might be a weapon with the tags: (3-harm far area loud reload). In expanded form, this weapon does 3 standard harm; it has a range of `far' only, so cannot be used closer; the 3-harm is done to an area (good prompt for the GM); using the weapon is loud; and it needs time to reload between uses (more prompts for the GM).  

The challenge of tags is two fold. The first is that they can range from extremely specific (3-harm) to intuitive but fuzzy (loud) to wooly (reload). How loud is loud? How long does it take to reload, can it be reloaded in a firefight or is that tag more about counting down limited and rare ammunition? Reloading an RPG launcher has different constraints to reloading a taser, and if it's actually the spells Summon Meteor vs Spiders Embrace, the different people at the table may have very different expectations about how long it takes to cast again. One solution is to make all tags very specific, preferably without bogging the game down, another is to differentiate between explicit rule tags and guidance tags. The intent on this is to avoid people looking things up if they aren't really covered in the rules. 

The second challenge of tags for the game designer is that they add a lot more semi-intuitive, semi-formal   words to the vocabulary a player needs to be aware of.  Since they are very easy to add to a game, and easily flow between wooly and formal between design drafts, you can end up with a lot of mildly inconsistent words being used. Be doubly careful they don't overlap with the flavour words you've used elsewhere. In a Space-Cowboy game perhaps +boring is used to tag challenges that are high risk but quite hard to concentrate on, like the day of calculations before a warp-jump, but the move ``Relativistic Dreamer'' has the trigger: when you daydream a new invention while doing some boring mundane chore, roll+Fusion''. It seems fair that all +boring tasks are boring, but also something that is not high risk, like mopping the cargo bay is boring but not +boring.
Why would you include the +boring rule in the game? Perhaps it's on theme and you are looking to capture that Space-Trucker working stiff vibe, or perhaps because you already included the +concentration tag and you wanted a mechanic to make it worth taking. Tags and rule interactions and complexity scale rapidly. They will probably cause you the most headaches in the `almost finished' drafts of the game. As with everything, they should support thematic play. The more of them on the sheet, the less impact each one will have on the game.\footnote{of course that might make them useful `small change' currency for player development. Much less powerful then a new move or stat boost, but more directly useful then an XP point}

They do have some great uses. The Veil, a cyberpunk rpg, embraces tags to separate types of damage, types of special implant effects and types of disadvantages. This detailed and clearly defined vocabulary supports the players in crafting their own implants/items, something that is very on theme. Apocalypse World does something similar to build strengths and weaknesses into gangs and hardholding, but mixes explicit rule type tags with evocative fictions ones that I, personally, find confusing at times.

City of Mists uses tags instead of stats. To punch someone you use the `Hit with all you've got move', and roll two dice and add all tags that you have relevant to punching someone. Some GMs have reported this drives players to spend a lot of time trying to justify why as many stats as possible apply to every single move. I worry (but have not experienced) that the consideration, and potential negotiation shifts the mind set, pulling you out of the game a little. A comparison can be drawn to the cerebral thrill of a good Fate game with many levels of rules and story awareness humming at once instead of Apocalypse World style deep immersion in cinematic events. If you avoid this by pinning the tags to moves too explicitly, you recreate stats.

City of Mists also allows negative tags (or, to be more precise, tags that are negative in that situation) to reduce your rolled total by one. This is compared to most PBTA games where negative tags are guidance for the GM in the event of a failure, but have no affect on the chance of failure occurring. As such accruing negative tags tends to have weak story impact but does follow the principal of "tell the consequences and ask". 

\section{Equipment List}

The size of this section will vary, depending on how much explaining you have to do. The more complicated the rules, and the more ways where players interact with the world through the medium of equipment, the longer this section will need to be to support the rest of the rule set.
This is the section of rules that has never really excited me. In writing a hack I would check whether equipment really matters, and if not, just compile the starting equipment from the playbooks and add equipment descriptions as the need comes up in playtesting. If equipment *really* matters, then I might start looking into something more structured then tags. 

The Sword, Crown and Unspeakable Power has a good minimal example. The setting is dark fantasy, so everyday technology doesn't need explaining. and the focus of the game is on scheming, not trading up equipment. There's a paragraph each for `Small Weapons, Large weapons, Ranged weapons, Arcane Weapons and Armour' that defines their size of effect in terms of harm(HP), a list of tags and definitions, and a short acknowledgement that `Other Gear' exists, but is even less important. 
Apocalypse World has more text. It covers the value of in game currency (one unit of barter is a month's hospitality or a night of luxury and company). It repeats the moves for finding and buying stuff to help frame the following pages in context of the theme of scarcity/availability. It defines all the tags that apply to things,\footnote{It also tags each tag with one or more categories: mechanical, constraint, cue. This is useful, but I still resent having to flip to the right part of the book to check wether *worn* and *implanted* have different mechanical effects.} As the genre demands, it goes into more detail on common weaponry, improvised and specialized. It repeats the definitions of each playbook's specialist gear, vehicles, prosthetics,  and then goes on into other things some playbooks start with and others may acquire: workspaces, holdings, gangs. It's duplicative, but helpful for people reading about the `world' rather than the individual playbooks.
The Sprawl, being a cyberpunk heist game goes into longer sections for common and specialist equipment. It has rules for you to cash in preparation tokens to have just the right bit of equipment when you need it, but that equipment stills needs defining, and a rebreather is the kind of equipment that is far enough away from shared experience that some GMs will find it helpful to have a written source to draw from or point to. The goal of these sorts of rules is to support the table having a shared experience, and avoid one person being thrown out by cognitive dissonance.

An example of really detailed equipment in a game is Flying Circus. In this game of `aviation fantasy', the players are the Pilot, but arguably, their planes are also characters and come with their own rules, detailed stat blocks and character sheet. In a way this embodies the free spirit vs grounded details paradox of that genre, but it does mean a different playstyle. 

\section{Clocks, Threats, Fronts, Storms, Maps, Bestiary, tables}
This is another of those sections that seems to vary from game to game. This section is about how you, as a designer, support the GM to create the kind of settings that result in the genre playing out. So it makes sense it would vary between genre, but when you look at the big changes between Apocalypse World editions, it seems clear that there is not yet a clear set of clear ways to do this.

\subsection{Clock}
The Sprawl has it easy, the mission structure of the game means that supporting the GM here is easy, and consists of table of mission types broken down into mix and match stages and step by step sample mission. The Sprawl runs on clocks, where you note out before hand the corporate response strategy to increasing security issues, so as the GM, you just have to follow your clocks (if this happens the corporation does this) and respond as the players do things. A big advantage to these is to keep the GM honest and it's less tiring. then constant improvisation. A disadvantage is that they can't easily take advantage of narrative fluidity in the session., and if the players decide to go somewhere new, you've got to create suitable threats on the fly anyway, and is there anything in the rules to help with that?

\subsection{Types of Threat and the Bestiary }

Apocalypse World, like most PBTA games, uses Threats. In this case it has seven kinds of threat, each with six sub-types corresponding to different impulses, moves and a bit of description. It's quite a lot to take in, and I'm not sure how well used they are compared to just creating NPCs in the same format as the playbooks and letting them loose on the world. The section is equivalent to a table of options.

Dungeon World, operating in the fantasy genre, has a Bestiary rather than Threats. This means that rather than the slightly abstract threat types and subtypes, there's just a long set of pages of different monsters, their instinct, armour, health, wealth, special qualities and relevant tags, all grouped by habitat. It's the same end table effect, but individual creations are graspable and usable at once, while still being clearly hackable. I think it might be a better way to provide that GM support, and has the benefit of helping convey the world you are imagining as a designer, to the GM.


\begin{itemize}
\item Threat type 1
\item Threat type 2
\item Threat type 3
\item Threat type 4
\item Threat type 5
\item Threat type 6
\item Threat type 7
\end{itemize}

\subsection{Fronts, Storms, Maps}

Urban Shadows takes the Threats concept and builds them into Storms (called Fronts in some other games). The idea of this is once you have one or two Threats, you look at what theme connects them and build more threats that tie into that. This means everything happening at the table is building towards that theme and drawing people into that specific shared experience, perhaps.

Apocalypse World second edition takes the Threat concept and arranges them by means of the Threat Map. This is an abstract map that is always centred on the players, and tracks what is getting close, what is circling and what `direction' that threat lies in. I love this approach, and reused it for both cyberpunk and fantasy games I was running, but I am a very visual learner. I know another GM who wrote a python program to generate and draw out the relationship network of people for a community based game. She needed the image to track the ripples of player behaviour.

So here is a bit of untested, probably contentious advice. Provide multiple ways to organise threats that correlate the different learning styles. This will let the GM choose a method that helps them remember what's going on.

\begin{itemize}
\item Visual: Drawn maps or networks.
\item Aural: There may be an equivalent to Fronts in musical chords, but you could also tie threats in a front to different roles in a band.
\item Verbal: Tags are the epitome of this
\item Physical: A bag of tokens or a hand of index cards to draw from.
\item Logical: A monster generator, perhaps a series of interlinked tables and lookups? 
\item Social: Threats defined by which player they scare and which will likely beat it
\end{itemize}





\section{Theme Deconstruction}\label{themedecon}
\begin{multicols}{2}
\subsection{balance}
+ GM Principle: meaningful player character decisions shift the balance
+ GM Move: Worst imbalance reduces but new imbalance develops.	
Possible mechanic: split playbook experience and moves into balance/imbalance tracks

\subsection{belief}
+ GM Principle: not all belief is true. sometimes the players are right
+ GM Move trade loss for benefit (in accordance with belief)
+ Possible mechanic: ritual elements to reinforce character will or add details to world

\subsection{change}
+ GM Principle: What has changed since the player's last visited. What hasn't?	
+ GM Move: Change one constraint on the situation, people or landscape
+ Possible Mechanic: push your luck for directed change of landscape or yourself ?

\subsection{comfort}
+ GM Principle: agree with players what comfort means. check in occasionally.
+GM Move: Throw a feast, party or concert
+ Possible Mechanic: meet npcs comfort level to gain ally

\subsection{destruction }
+ GM Principle: First consider destroying npcs and your MC ideas	
+ GM Move: "trade harm for harm (as established) or deal harm (as established)"	
+ Possible mechanic: Detailed harm tags (messy, knockback, seige ect)


\subsection{corruption}
+GM Principle: corruption requires the player's character to feel guilt over their actions	
+GM Move:  offer a conflict of interest. both good	
+ Possible Mechanic: Corruption moves that can be activated on a spefific trigger

\subsection{family}
+GM Principle: every parent was once a child
+GM Move: 	announce a flashback
+ Possible Mechanic: 	inheritance of soft moves

\subsection{freedom}
+GM Principle:  There is always another place to run to
+GM Move:  ask a silent player to describe location ahead.
+ Possible Mechanic: expansion of vehicles into mobile bases 

\subsection{friendship}
+GM Principle: friendship spreads through interlocking circles cemented by favours
+GM Move: an npc calls in a favour. reputation cost to say no 
+ Possible Mechanic: do favours, receive debts. call in for favours
		
\subsection{glory}
+GM Principle: wherever they go, PCs should hear of great adventurers. sometimes, it might even be them
+GM Move: create a competition for the players
+ Possible Mechanic: gained by beating more difficult challenges and spent to create more difficult challenges  

\subsection{honour}
+GM Principle: always ask if an npc action is honourable. justify it
+GM Move: deal harm to players honour, justified or sneaky
+ Possible Mechanic: honour only applies to certain ranks. it means that oaths carry weight but only for those ranks.

\subsection{hunger}
+GM Principle:  always describe food, fatness, health first
+GM Move:   reveal an upcoming scarcity
+ Possible Mechanic: needs and consequences tag system

\subsection{justice}
+GM Principle: What typifies this situation: the scales, the blindfold or the sword?	
+GM Move: trade judgemet, compassion or violence (as established)
+ Possible Mechanic: ask the abyss "is this just?" Mark bonuses at end of session 

\subsection{knowledge }
+GM Principle: knowledge is the lever that player characters can use to magnify their actions	
+GM Move: trade knowledge for knowledge (as established)
+ Possible Mechanic: cash in knowledge for bonuses on quadratic scale

\subsection{law/tradition}
+GM Principle: everybody, everything has a couple of freely known behaviours
+GM Move: Invoke tradition to create a new threat
+ Possible Mechanic: two polar stats to make quadrant block between four traditions. 

\subsection{love}
+GM Principle:  love is steadily rising level of intimacy
+GM Move: spotlight a connection or triangle
+ Possible Mechanic: intimacy moves: 	 "i'd like to X, may I?", "yes but"

\subsection{pain}
+GM Principle: Scars are not just physical. Damage is not just permanent
+GM Move: deal pain, as established
+ Possible Mechanic: memory/pain as addiction 

\subsection{revenge}

+GM Principle: NPCs will always be avenged by someone
+GM Move:  offer vendetta, offer ritual peace at cost
+ Possible Mechanic: detailed followers and bonds rules. fail to avenge one, loose others due to disgrace.

\subsection{unity}
+GM Principle: always introduce someone by their unit/faction/clan before any other detail	
+GM Move: create player-npc-unit triangle over a resource 	
+ Possible Mechanic: Faction tags on all players and NPcs that can be used as PC move triggers


\subsection{survival}
+GM Principle: respond with fuckery and intermittent rewards
+GM Move:  offer something, but make them roll the dice
+ Possible Mechanic: Crafting guidelines for making better stuff from crappy parts

\subsection{violence}
+GM Principle: Sometimes the direct solution is the correct one
+GM Move: 	add violence not directed at pcs
+ Possible Mechanic: conflict escalation mechanic. disagreement to posturing to threats to violence 

\subsection{wealth}
+GM Principle: currency is liquid. where does it flow?
+GM Move: replace a source of wealth with another	
+ possible Mechanic: investment schemes as missions 

\end{multicols}

\newpage

\section{Skeleton World - the (example) game.}



\subsection{Decisions:}

\smallcaps{Scale:} Simplicity. I want a simple game, scaled back where I can. \sidenote{This is because it is an example, so the less I have to write the better as it will make the decisions and structure of the example clear. There is also a thematic unity between playing as skeletons and a bare-bones game}

\smallcaps{Key Imagery:} Coffins `poofing' into existence in the sky and falling into a bone white desert in cloud of dust. A sword wielding skeleton emerging from the ground; cartoon skeleton's in a danse macarbe (or a grim fandango); a burning skeleton throwing fireballs, a knight in armour made from hundred of ribs; a mishapen skeletal beast with too many limbs and joints that bend backwards, a screaming skull floating, a skeleton in a suit backing away in horror as plants consume another.

\smallcaps{Community vs Party:} Adversarial. I want to chase out the bleak loneliness of death and the grim-darkness of being a skeleton, trapped in a timeless world, being slowly ground down against each other. \sidenote{this was a key decision. A game about a special forces unit in a Necromancer's Army would be very different. And possibly more fun?} 

\smallcaps{Key Theme:} Change. The world and environment are unstable and shifting. \footnote{this was chosen partly by default as non of my other examples in the Theme Deconstruction section seemed to fit.}
\smallcaps{Secondary Theme:} Scarcity. The world is basically a wasteland with only a few key items belonging to the players. This combined with Change suggest dunes blowing. What makes the sand? The ground down bones of ancient skeletons of course!
\smallcaps{Palette:} Emphasis on bones and the limited things the skeletons bring through from the living world.

\smallcaps{Statistics:} Lets go for a simple variant on mind, body, spirit as BONES, MEMORY, GLUE
\smallcaps{Experience:} Let's go for a simple method. At the start of each session, each player gives a brief recap of what they did and marks a new move. No experience is tracked.
\smallcaps{Playbooks:} The Key imagery suggests a few already. Two points spread across three stats gives six unique statlines, lets use them as the basis. Let's give each a unique glue to add a bit of imagery to split them up, and lets assign each playbook two of St John Ross's 12 RPG archetypes
\footnote{These are taken from the Risus Companion. I heavily recommend buying it, especially if you are curious about GMing tips for laughter-inducing, surprisingly deep games}
, it will give the playbooks a bit more depth.

\begin{itemize}
\item BONES 0, MEMORY 2, GLUE 0  
\myitemend Splint+clothes  	--	Persuading			--		Gadgeteer
\item BONES 1, MEMORY 1, GLUE 0  
\myitemend Beast+wires			--Wilderness survival  --		Intrusion
\item BONES 0, MEMORY 1, GLUE 1  
\myitemend Mage+fire				--Communication+Protocol--	Scholarship
\item BONES 2, MEMORY 0, GLUE 0  
\myitemend Knuckles+scrimshaw  --Athletics		--			Drive/Ride
\item BONES 1, MEMORY 0, GLUE 1  
\myitemend Glom+guts			--	Medical		--				Combat
\item BONES 0, MEMORY 0, GLUE 2  
\myitemend Skull+plant			--	Detection	--				Weird
\end{itemize}

\subsection{Basic Moves}

\smallcaps{Struggle}
When you struggle for control of something, roll+Bones.\sidenote{So many decisions here. First is that since there are only three stats I want to limit it to three thematic ways to take action, the trigger here covers combat, but also argument, intimidation and person versus an uncaring universe. Second is that struggling for control is dangerous, so damage is delivered on nearly all options. Third is that, for the basic moves at least, I'll prescribe a 6- result to help the players expectations stay in genre. Fourth is the introduction of a new mechanic, just adding a little extra spice to the dice roll. With only three moves I don't want them to feel too samey in their results.}\\
On a 10+ exchange damage and take control of the thing\\
On a 7-9 exchange damage, take control of the thing and give up a position, something valued or an honest answer to your attacker.\sidenote{you may feel that an `honest answer' is the obvious least-cost choice, but it acts as a good way for the MC to inject loaded questions to draw out characters. The other options are protection for players who hate being put on the spot like that.}
\\ On a 6- the result is as 7-9 but you don't get control.
\\If you also roll a double on any result, don't take damage.
\\

\smallcaps{Unleash} \sidenote{can't have a limbo based game without good and bad memories showing up, and this is a great place for the change theme to be contrasted. Name reduction is a type of HP/manna clock and seemed a thematic way to avoid memory spamming. At this point in the design I'm not sure if I will allow a way to recharge your name with fresh letters, or if the pacing should be to allow one use per session (on average) for a six session campaign. }
 When you unleash a memory to try and shape something out of the dunes Roll+Memory
\\On a 10+ it is there and will not change, ask the MC how many letters to reduce your name by
\\On a 7-9 as 10+ but it is weak, temporary or wild
\\On a 6- something bad is unleashed instead. It will haunt you.
\\
\smallcaps{Take Damage} \sidenote{ a little bit of randomisation which means that weapon descriptions can have fixed damage. It is primarily to cover the genre image of a skeleton pulling itself back together from a thousand pieces. The suitable repair is supposed to generate a hook/pressure for the player character to take action.}
 When you take damage, even zero damage after armour, Roll+Glue
\\ On a 10+ it's not so bad, take the damage but reduce it by one or take take the 7-9 option.
\\ On a 7-9 take the damage OR ignore damage but reduce Glue to zero until you suitably repair it (see playbook)
\\ On a 6- take the damage. Your armour fails to do anything.  OR ignore damage but reduce Glue to zero until you suitably repair it (see playbook)

\smallcaps{Decision:} No move for helping/hindering another player. It's a struggle out there. 

So that is three stats that have led to three basic moves (and three little extra mechanisms - rolling doubles, reducing name length, repairing Glue). That is ok for now, but I hope to tighten that up with a bit of playtesting.

\newpage
\subsection{Playbook: Splint}
BONES 0, MEMORY 2, GLUE 0  
Name - six letters long
\\ Glue - clothes . Repair by sewing, patching (if you can find the needle and thread) or replacing items of clothing (if you can take them). 		
Your job in life is mixed up in your identity, such that some of your bones are replaced by items symbolic to you. 
\\
\smallcaps{Start Move}: Start with two implements (choose anything from the equipment list)\footnote{Equipment list: Wrench, till recipet roll; typewriter; chef knife; teapot; stethoscope; pens pens pens; mop; hard hat and boots; camera; measuring tape; material shears; every car key; globe}. Implements are always available, but cannot be removed or given to others. Say what body part they replace.
\begin{itemize}
\item  Junkyard: detail another implement that is now part of you
\item  Substitute: you can `consume' a suitable item to heal three damage
\item  Donate (needs Substitute): Spend three health to give anyone a previously substituted item
\item My Uniform: Increase Memory by +1 permanently
\item  Strangely Dishonest: When you lie to their face Roll+Memory. 
\myitem On a 10+ they buy it, and it may even change the world. 
\myitemend On a 7-9 they buy it for now.
\item Presence: When you unleash a memory, you can say who, if present, is unable to look away. 
\item Demagouge: When you try to lead a volatile crowd Roll+Memory
\myitem On a 10+ if you meet their needs, they'll meet yours
\myitemend On a 7-9 as 10+, but strictly in that order!
\end{itemize}


\newpage
\subsection{Playbook: Beast}
BONES 1, MEMORY 1, GLUE 0  
Name: Three letters long
\\Glue: Thin brass wires. Repair by scavenging the wires from broken or moving dune monsters.
You've been here a long time, if time even passes here. There are many that have entered here before the `humans', and you are cunning enough to have survived.

\smallcaps{Start with one of:} Big Teeth (+3 damage), Huge size (+2 armour), tattered wings (limited flight)
\begin{itemize}
\item  Lamarkian: Take another starting thing. Duplication allowed.
\item  Reputation: Add four more letters to your name
\item  Cunning: When you hide in the dunes Roll+Mem
\myitem On a 10+ you are undetectable until you decide to act
\myitemend On a 7-9 you are invisible to common appearances
\item  Shoving: When you struggle with another beast ritually over territory or similar, on a 6- you won't exchange damage.
%\item  Pounce: When you commit to pin someone down, roll+Bones
%On a 10+ they are trapped
%On a 7-9 you take damage but they are trapped.
%On a doubles, don't take damage. \sidenote{this needs to be added to avoid the move being worse then straightforward struggle. It dosen't add anything to the move. Perhaps the double mechanic should go, or Pounce is simply not different enough from Struggle to justify existence}
\item  Cage Breaker: When you pick a lock with your long wire tounge, Roll+Glue
\myitem On a 10+ it is delicate and quick
\myitemend On a 7-9 it is neither, but you get the lock open
\item  Call of the Wild: When faced with an constructed obstacle you break or bypass Roll+Bone
\myitem On a 10+ you succeed or the MC suggests an alternative good route
\myitemend On a 7-9 you succeed. Enjoy your new situation! (ask the MC)
\end{itemize}


\newpage
\subsection{Playbook: Mage}
BONES 0, MEMORY 1, GLUE 1  
Name: five letters long
\\Glue: coloured fire. Repair by lighting and building a small fire, and bathing in it.
Obsessed or just passionate, you can't see why everyone doesn't agree with you.
\smallcaps{Start with Projection}: You scoop and throw a fireball of desire dealing 2 damage.
\begin{itemize}
\item  Griefer: You pull a blinding cloud of ash and embers out of the ground around you.
\item  Furnace When you crush someone into your rib hollow Roll+Glue
\myitem On a 10+ do 10 damage to them
\myitemend On a 7-9 as 10+ but you reduce Glue to zero until repaired
\item  Hatred: When you spit your fire on someone Roll+Glue:
\myitem On a 10+ it clings and continue to burn and damage them until you speak to them
\myitemend On a 7-9 as 10+ but strangely, although it burns it doesn't actually damage them.
\item  Scholar: When you study something carefully Roll+Memory
\myitem On a 10+ you spot something no-one else would. Say or ask the GM
\myitemend On a  7-9 you have an idea about it. Ask the GM
\item  Enlighten: When you are stifled by rules Roll+Glue
\myitem On a 10+ you guess who to challenge to unmake the system
\myitemend On a 7-9 you guess where the power lies
\item  Flare: When you close your eyes you can appear as a small fiery image to anyone you have previously damaged. This power may reach over distance and through walls.
\end{itemize}

\newpage
\subsection{Playbook: Knuckles}
BONES 2, MEMORY 0, GLUE 0  
Name = eight letters
\\Fire: Scrimshaw. Repair by having an artist add scrimshaw to replacement clean bones.
The simplest approach is often the most direct.
\smallcaps{Start with: Brawler}:Even unarmed you still do 3 damage
\begin{itemize}
\item  Bullish: When you hit someone hard you can choose to knock them backwards
\item  Naturally Spiky: Reduce damage taken by 1. This does no stack with armour.
\item  Athlete: When you push to do something others can't Roll+Bones
\myitem On 10+ you do it.
\myitemend On 7-9 you do it but take 1 damage.
\item  Artistic: As long as you have a working hand, you can record your feats in scrimshaw (counts for repair glue). Otherwise you must find an artist.
\item  Domination: When you wrestle a large dune monster into submission Roll+Bones \sidenote{struggle theme again -  no easy loyal animal companions here!}
\myitem On a 10+ it accepts you. Treat it well and it will let you ride it.
\myitemend On a 7-9 as 10+ but it dislikes you and will use every opportunity to escape.
\item  Martial Dancer: When you prepare to fight Roll+Mem
\myitem On 10+ hold 3,
\myitem On 7-9 hold 1,
\myitemend During the fight, spend hold one for one to - Move unstoppably; break a grip; block an attace\end{itemize}


\newpage
\subsection{Playbook: Glom}
BONES 1, MEMORY 0, GLUE 1  
Name: six letters
\\ Glue: Guts. Repair by consuming fresh meat or funeral foods.
You always blended in well, and bones are bones are bones right?

\smallcaps{Start with: Amalgamate}: When you gut wrap sufficient new bones into any skeleton Roll+ Glue: 
\begin{itemize}
\myitem On a 10+ heal 3 or heal someone else 3 
\myitemend On a 7-9 as 10+ but reduce glue to zero until repaired
\item  Gall Stones: You develop a pot belly stuffed with bones and sand to digest even old bones. Consuming a full skeleton is enough for you to repair your glue.
\item  Blemeye: Your skull moves down inside your ribcage. Increase you base armour by 2. More armour stacks with this.
\item  Cage and Club: One arm ends in a ribcage shield, the other in another femur. You always count as armed with a Large Club, and the shield increases your base armour by 1. 
\item  Gut Lasso: Needs Gall Stones or Cage and Club. When you throw (or spit) a lasso made of your guts, roll+Glue.
\myitem On a 10+ you get them. You can unbalance, disarm them or draw them closer.
\myitemend On a 7-9 you get them but it's awkward. Choose as 10+ but then they choose an option to do to you.
\item  Double Jointed: Needs Cage and Club or Blemeye: You gain a second pair of arms. If this means you have four hands this allows you to duel wield Great weapons. Otherwise it just means you have hands free to carry things.
\end{itemize}

\newpage
\subsection{Playbook: Skull}
BONES 0, MEMORY 0, GLUE 2  
Name: eight letters 
\\ Glue: Plant. Repair by filling skull with water and staying in the sun for a short time.
In a world of white sands, what is left of you brings the true death. 

\smallcaps{Start with: Manifest:} When you need to be more then a slowly floating skull Roll+Glue
\\On a 10+ you manifest a knotted viney body, as strong and fast as any skeleton.
\\On a 7-9 Choose one: you are small, limbless or slow
\begin{itemize}
\item  Rooted: When you appear to remain motionless, you may push a limb out of the dust three body lengths away.
\item  Infectious growth. When you deal damage and roll doubles you can choose to ignore their damage or to infect them. Infected take one damage each turn ignoring armour.
\item  Tendrils: No-one can sneak up of you. Your attacks can entangle.
\item  Blossom: When you exert power through an infected, they blossom with large flowers. They become the focus of the attention of anyone around.
\item  Detection: 
\item  Weird:
\end{itemize}

More work to be done here I'm afraid!


%\section{Page Layout}\label{sec:page-layout}
%\subsection{Headings}\label{sec:headings}
%This style provides \textsc{a}- and \textsc{b}-heads (that is,
%\Verb|\section| and \Verb|\subsection|), demonstrated above.

%The Tufte-\LaTeX\ classes will emit an error if you try to use
%\linebreak\Verb|\subsubsection| and smaller headings.

% let's start a new thought -- a new section
%\newthought{In his later books},\cite{Tufte2006} Tufte

\bibliography{Skeleton_world}
\bibliographystyle{plainnat}



\end{document}
