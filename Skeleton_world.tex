\documentclass{tufte-handout}

\title{An example of world and game making for Powered By the Apocalypse\thanks{Inspired by the amazing SimpleWorld of Avery Alder}}

\author[Cromlyn Games]{Cromlyn Games}

%\date{28 March 2010} % without \date command, current date is supplied

%\geometry{showframe} % display margins for debugging page layout

\usepackage{graphicx} % allow embedded images
  \setkeys{Gin}{width=\linewidth,totalheight=\textheight,keepaspectratio}
  \graphicspath{{graphics/}} % set of paths to search for images
\usepackage{amsmath}  % extended mathematics
\usepackage{booktabs} % book-quality tables
\usepackage{units}    % non-stacked fractions and better unit spacing
\usepackage{multicol} % multiple column layout facilities
\usepackage{lipsum}   % filler text
\usepackage{fancyvrb} % extended verbatim environments
  \fvset{fontsize=\normalsize}% default font size for fancy-verbatim environments

%for fancy lists
\usepackage{tikz}
\usetikzlibrary{shadows}
\newcommand{\mylist}{\tikz[overlay]\draw(-.2,-.2)--(-.2,.5) [path fading=east](-.2,.15)--(.1,.15);} %adds the |- shape to the start of each list item
\newcommand{\mylistend}{\tikz[overlay]\draw(-.2,.15)--(-.2,.5) [path fading=east](-.2,.15)--(.1,.15);} %adds the |- shape to the start of each list item
\newcommand{\myitem}{\item[\mylist]} %defines the scope of the mylist command to be 2nd level sublists
\newcommand{\myitemend}{\item[\mylistend]} %defines the scope of the mylist command to be 2nd level sublists

% Typesets the font size, leading, and measure in the form of 10/12x26 pc.
\newcommand{\measure}[3]{#1/#2$\times$\unit[#3]{pc}}

% Macros for typesetting the documentation

% Generates the index
\usepackage{imakeidx}
%\makeindex[name=moves, title={Index of moves}] % on reflection, not needed. 
\makeindex[name=stuff, title ={Index of elements, items}]
\makeindex % general index for playbooks and stuff



% Standardize command font styles and environments
\newcommand{\doccmd}[1]{\texttt{\textbackslash#1}}% command name -- adds backslash automatically
\newcommand{\docopt}[1]{\ensuremath{\langle}\textrm{\textit{#1}}\ensuremath{\rangle}}% optional command argument
\newcommand{\docarg}[1]{\textrm{\textit{#1}}}% (required) command argument
\newcommand{\docenv}[1]{\textsf{#1}}% environment name
\newcommand{\docpkg}[1]{\texttt{#1}}% package name
\newcommand{\doccls}[1]{\texttt{#1}}% document class name
\newcommand{\docclsopt}[1]{\texttt{#1}}% document class option name
\newenvironment{docspec}{\begin{quote}\noindent}{\end{quote}}% command specification environment

\begin{document}

\maketitle% this prints the handout title, author, and date

\begin{abstract}
\noindent
This document describes some ways to make  a Powered By the Apocalypse game.
It should support reskins, GM's looking for setting ideas, and help building a custom class.
It also supports deep hacks that play with the fundamentals of the game.
\end{abstract}

%\printclassoptions
There's three ways to use this document. One is around a table and build the game the table wants to play together. Another is to pick and choose bits from it to help you explore an idea. I like to pick randomly at each stage, forcing just the right amount of constraints that there can only be one solution for each thing from stat to archetype to individual move, taking me outside where I can get normally.

\section{Size}
Choose one:\\
Tight game: one themed additional mechanic\footnote{An example is Apocalypse World, which ties everything to destruction} \\
Balanced tension game: two themed additional mechanics\footnote{An example is Urban Shadows, which introduces Debts and Corruption}\\
Focused big game: one themed additional mechanic, second theme present as a stat.\footnote{An example is the Sprawl, which adds synth as a stat for cyberwear}\\


You need a fast, nuanced strong hit / weak hit / fail generator\footnote{The standard method is two six- sided dice with margins set at -6, 7-9 and 10+. This can be easily mapped to other types of dice. Cards are slower but can work. Dream Askew used tokens. I'm intrigued by rock/paper/scissors but there are game flow implications}
 and a theme and a genre. This will give you an Agenda and Principles. These will give you GM moves. 

Choose at least two and normally four or five:
Stats, Basic Moves, Playbooks, Development, Scenario Moves, Inserts, Tags.

\section{Genre}
What genre (or mashup) do you all want to play in?
You may know, if not pick one or two from each below:
\begin{multicols}{3}
ancient legends\\
high fantasy\\
low fantasy\\
grim fantasy\\
age of sail\\
mediaeval\\
pike and musket\\
steampunk\\
Victoriana\\
colonial\\
1920s pulp\\
superhero comic\\
weird world war\\
biopunk\\
cold war\\
diesel punk\\
urban fantasy\\
slice-of-life\\
alternate history\\
dystopia\\
hard sci fi\\
solarpunk\\
cyberpunk\\
post-apocalypse\\
space opera\\
\end{multicols}

\subsection{(Optional) Party is}
\begin{multicols}{3}
circus\\
first responders\\
mercenaries\\
city gaurds\\
criminal gang\\
merchant convoy\\
monster hunters\\
smugglers\\
traders\\
investigators\\
boarding school\\
university dept\\
neighbours\\
village\\
operatives\\
rebels\\
police\\
refugees\\
outsiders\\
beaureucrats\\
\end{multicols}

\section{Theme}
What feeling or thing is this game about?
Choose one or two, eg
"What would you do for X"
or
"Is X more important then Y?"

\begin{multicols}{3}
balance\\
belief\\
change\\
comfort\\
destruction \\
corruption\\
family \\
freedom\\
friendship\\
glory\\
honour\\
hunger\\
justice\\
knowledge \\
law/tradition\\
love\\
pain\\
revenge\\
unity\\
survival\\
violence\\
wealth\\
.......\\
.......\\
\end{multicols}


If you aren't sure, take a read of the Theme Deconstruction Section at the end of the document.

\section{Stats}
Aim for the fewest number you need. Keep them simple, memorable, and immediately highlighting the difference in how different characters approach challenges. A good example is Versed, Young, Gendered, Wyrd from Sagas of the Icelanders. It clearly spells out what is the important differences between two characters.

Choose one set and rename for your genre, or mix and match between sets:\\
1. Fighter, Thief, Wizard\\
2 .Social Circles, Danger Triangles, True Squares, Stitched Crosses \\
3. Assertive, Persuasive, Curious, Methodical, Confident\footnote{this one is loosely based on the Big Five Personality model}\\

Or choose them by Genre, either on skills, common environments or to support basic skills you expect all players to need:\\ 
a. Space, Robots, Lasers,  Feelings\\
b. Honour, Rice, Ki\\
c. Fate, Power, Control \\

\subsection{Allocating stats}
WIP


\subsection{If no stats}
If you are not having stats, there's three ways to  handle it\\
1. Just roll and rely on the nature of the moves for variety.\\
2. Mark each move for basic/good/advanced. Basically treat each move as it's own stat, similar to a skill based game engine. Example  {Exo-adventurers or the John Harper hack that isn't called Blades in the Portals}\\
3. Use ephemeral stats like Roll+bodies searched or Roll+Hands free. Be careful with this. If it is going to be tracked from scene to scene (like number of eyes) just make it a stat and let them write it on their character sheet.

\section{Basic Moves}
Ideally start players with seven or less moves to remember. I tend to split it as five basic moves, two playbook moves. I tend to have one basic move per stat. The core of a basic move should be "Take thematic action"\footnote{EG Apocalypse World "Open your brain" to the maelstrom to shift the status quo, while Urban Shadows has you "let it out", risking probable corruption in return for supernatural power}. Examples of thematic basic moves are offered in the Theme deconstruction sections.
All Basic moves should be thematic. They should also support play. Often a player character will want a genre (not generic) version of:\\
a. Get information\\
b. Persuade/seduce\\
c. Help/Interfere with another player\\
d. Buy/get things\\
e. Mortal peril \footnote{well, they might not want it, but they may need it}\\

\subsection{Get information}
This is often in the format of roll+stat and get hold, spend hold to ask questions. The use of hold allows players to develop lines of inquiry as more information becomes available in the scene, rather then the distraction of forcing rerolling many times. 
The questions available to be asked are often listed and the choice of what can be asked is important, it tells the player what their character thinks is important about the genre. The Sprawl, ( mission based cyber-punk), has "Read A Situation", Urban Shadows (political urban fantasy) has "Figure Someone Out".
Players will act on the information they get. If you make them ask thematic questions they often end up taking thematic action.

\subsection{Persuade/Seduce}
The challenge with this move is  balancing the requirements of the genre/theme (in a Noir game, the Femme Fatale must be able to manipulate the other archetypes) and , zooming out a little, balance the social contract of the table and the expectation that players have of their own agency. Storyteller-type players might be happy to sacrifice character agency in return for the ability to add plot twists or elements of world building. Actor-type players might be happy to give up all power outside their character's internal thoughts and decisions, but will hate anything that intrudes on the territory that they do control.

There are a few options:\\
1) Make the move apply to NPCs only. Get players to negotiate if their characters act on each other. This is only going to be ok in games where the player characters are supposed to be aligned teamworkers and where persuasion isn't a big thing in genre, like Fellowship\footnote{citation needed}. \\
2) Make two moves, one for manipulating NPCs and one for manipulating PCs. This is typically set as a standard move for NPCs and offers an extra carrot/stick for PC players to go along with your plan. This can work well for cinematic games where the player may want to get their character into thematic trouble.The Regiment has a move called "Impose Will" which uses only stick, but who the hell plays an Army themed game and doesn't expect to need to take orders?\\
3) Make one move, ignore PC agency. This will earn you some hate-mail, but can work well in games where players have other ways to influence the story then just their player actions. 

\subsection{Help/Interfere with another player}
The basic form is "When you help someone or hinder them, roll your relationship stat with that person. On a 10+ add or subtract 2 from their roll. On a 7-9, the MC will name a cost, if you accept this cost, then add or subtract two from their roll.

The exact nature of the relationship stat shifts between games, normally plugging into other mechanics. In Apocalypse World the Hx stat is constantly changing and is a source of character advancement. In the Sprawl the Links stat is mostly static, but sets up the initial threat clocks of different mega-corps. In Urban Shadows it is faction based, but also driven by the powerful Debt moves.

In a PBTA game, I think it is best to avoid allowing players to 'defend' against each other by making opposing rolls. It makes the format of moves break down and some complex snarls in the fiction to develop which distract from moving the story forwards. It's like the tennis ball hitting the net instead of being passed back and forth in a volley. Instead, the help/interfere move of the second player alters the chance of the first player changing the fiction the way they want to.

\subsection{Buy/get things}
I have included it here, because most PBTA rulebooks do address it, but as story driven themed games, shopping and equipment and gear tends to be of secondary importance. Examples of thematically addressing it are The Sprawl's move for getting additional cyberware installed where you have to choose between dangerous back alley docs or villainous megacorps. 
Apocalypse World has a barter move to cover sourcing 'some particular thing where it's not obvious you should be able to go buy it just like that", as a way to convey the Scarcity theme of the world, but allowing the players to perhaps drive the story further than a blanket 'no, you can't buy that.'. 
Both moves are all about giving players what they want, perhaps at a cost, and giving the GM some more hooks or levers to pull to put the characters into more interesting situations.

\subsection{Mortal Peril}
There are two branches to this move. The first is a catch-all "Take action under pressure" move that can be the main point of weakness in a game if the GM is forced to rely on it too much. 
The second branch is the move that kicks in should a player character actually die. It tends not to be something that comes up often in pbta games but when it does it should be important to the story, and the rules often support a gm making a big deal of it. Urban Shadows gives each playbook it's own 'in case of death' move. Dungeon World gives the player the option to become 'Death's Own', accepting that that character WILL die, but at the climax of a particular narrative arc.
The common basic move for 'running out of health' allows you to choose between taking a new character, changing playbook or taking a permanent hit to your character ( normally a stat reduction). 

\section{Agenda}
The GM Agenda is always:

\begin{itemize}
\item Play to find out what happens
\item Don't waste your players time
\item Deliver the world according to the genre setting
\item Deliver action in line with the theme
\end{itemize}

In other words, don't railroad your players but do respect the logic of the story world, the fiction. Don't pre-write a story that ignores them effecting the world and don't space out or make them grind for the cool moments `just because it feels like they haven't worked hard enough.' It can be a weird temptation as a GM, perhaps because the creativity required can feel tiring and you just want them to roll grinding dice checks for 15 min while you take a break. We'll cover techniques to help avoid that tiring feeling in the GM principles section.

\section{GM Principles}
\begin{itemize}
\myitem sprinkle evocative details everywhere 
\myitem make the world seem real
\myitem name everyone, make everyone human
\myitemend build a bigger world through play
\myitem create interesting dilemmas not interesting plots
\myitem address yourself to the characters not players
\myitem be a fan of the players characters
\myitemend destroy your creations, don't protect them.
\myitem make your move but misdirect
\myitem make your move, but never speak its name
\myitem ask provocative questions and build on the answers 
\myitem sometimes, reflect a question back upon the players 
\myitemend think off screen too
\myitemend + Additional from the theme (see Theme Deconstruction section)
\end{itemize}
Rewrite these to suit your genre. An example is "Barf Forth Apocalyptica" replacing "sprinkle evocative details everywhere." The word choice isn't critical, but you want to tone of the writing to help get the GM in the mood.\footnote{"getting you in the mood" is why I think Apocalypse  World calls the Games Master the Master Of Ceremonies. You are not about controlling the situation. You are about introducing each player character as an act that gets their own time in the spotlight and keeping the show moving.}

It may seem weak that the GM principles are so similar between different genre games and experiences, but that's becuase the first four are about world building, the next four are about managing your ego (MC, not GM!) and the last five are about ways to improvise that aren't as tiring as thinking up a plausible monster biology every five minutes. The final one of course, will vary with theme.

I am not going to go into a deep explanation of them all here, as I am kind of assuming you've played a few PBTA games before trying to write one. If you really need help, buy and read Apocalypse World as it has a strong guide with many examples of getting it right, or how to correct if you feel you have got it wrong.

\section{GM moves}
\begin{itemize}
\myitem put the spotlight on someone
\myitem seperate them
\myitem put them together 
\myitemend make their lives complicated now
\myitem give them a difficult decision to make
\myitem offer an opportunity, with or without a cost
\myitemend offer stuff that's painfully expensive but good
\myitem tell them the possible consequences and ask
\myitem turn a failed move back on them
\myitem use up their resources 
\myitemend activate stuff's bad side
\myitem use a front or threat move
\myitemend + Additional from the theme (see Theme Deconstruction section)
\end{itemize}

In the same way, I am not going into detail the GM moves or how to use the standard ones. The first four moves are about ways to move player characters about. There's no reason to have them all in the same place or working in perfect agreement all the time. Think of the beautiful jump-cuts of a heist movie where a separated team work together versus the intimate squabbling of a sitcom or the rolling partial scenes of a Shakespearean Play. It's something that many players have been trained out of or picked from cultural osmosis (never split the party!), and the onus is on you to shuffle them around to keep their characters uncomfortable and the players grinning.
The next four moves are all about giving your player's agency, opportunities to make meaningful decisions. My favourite one I eve managed was a cyberpunk game where the player was hanging off a crashed skyrail. ``You see someone tumbling towards you, that punk with a hook for a hand. You can try to grab him as he goes past , but you can't quite be sure which of his flailing hands you'll get. Otherwise he's a deadman. What do you do?'

This has obvious overlaps too with the next four items: setting up clear consequences. Generally, players won't mind horrific consequences if they knew the risk going in. ``you fail to close the portal, in fact it rips wide open, it's now far to wide for just the three of you to stop anyone coming out.'' In this vein, reminding them that stuff has a abd side makes it seem less arbitrary when you activate it. ``If you use grenades here, you risk destabilising the whole reactor.'' This also blurs back into ``Tell the consequences and ask''.  The level of overlap here is your friend, don't get to worried about the specific move. Just stab your finger on the list somewhere and do that.

The last two items are going to be more specific to your story, genre, setting and theme. They are in fact your primary ways as a GM (rather then game designer) to bring the theme out.  Even if the enemy is as simple as a Giant Rat, the threat move it comes with should be different in a cyberpunk game and a low-fantasy one. What should each be?

\section{Playbooks}
I normally aim for 6 to 12 playbooks, or do without them altogether. If you do that, then you can either let  players pick from a common list of moves (Class Warfare for Dungeon World), have only shared basic moves (World of Dungeons), or use only custom moves awarded in response to player actions in that session or the previous session (Experimental, but could work?).

Generally speaking I advise using playbooks. As a designer they make it easier to balance characters and helps keep the game more genre aligned and thus produce stories that feel like the table is working together to tell it. For the player, a playbook lessens the amount of information to process before making those first choices and gives you an easy to understand initial role in the story. Confident players make moves.
Why six to twelve? You want to give the last player to choose some choice, especially if they (or you) don't like two or more of the same playbook. I suggest you aim for everyone to have different playbooks as it feels more `special' for the player to have something no-one else can do. That connection is the start of the dice rolls meaning stakes.
If you are making playbooks then there's a bunch of ways you can space them out.  The first is by writing down archetypes of the genre (so in Film Noir we have the Detective, the Mobster, the Suit, the Femme, the Rival Cop, and then, playing on the genre time period and theme of guilt, maybe the Priest, the War Hero, the Waif, the Outsider). 

Next I'll start sketching out the character creation bits before getting too hung up on moves. 
As a player, I like the choose an option list approach in playbooks, as they tend to keep me on theme and provide seeds to get started quickly. My current recepie for these is five finished options, three rule based options and one blank to allow those who really want to to fill it in themselves (giving them a sense of ownership). \footnote{For example. Choose a Name: Fizzok, Hardcrust, Bluebell, Edbanger, Sappo, a flowery name, a punky name, a stolen name, \_\_\_\_\_\_\_}  The type of things chosen don't have to match between playbooks. Generally speaking I recommend following Vincent and Megeuy Baker's approach in Apocalypse World and basing phraseology around `appearance' and not `actuality'. 

Give people some starting equipment, and questions to ask other players. It helps tell those other players a lot about this one, and the answers can be used to get things moving from the start.


\subsection{sketching playbooks mechanically}
I like to do that as a first sketch. If I am using stats I'll write down a list of primary and secondary stats (eg Aa Ab Ac, Ba, Bb, Bc, Ca Cb Cc) and try to fix archetypes to it and fill the gaps. If you treat them as primary and secondary, then for N stats you get N squared slots (three stats is nine playbooks, four stats is sixteen). If you drop doubles that takes you to six and twelve respectively. If you keep doubles but instead treat Ab and Ba as the same then you get six and ten playbooks respectively (4+3+2+1). It's satisfying lonely fun. 

If I have other mechanics live in the game I might do the first sketch list with that. Cyber\_peripherals, for example, is about Gangs, Holdings, Favours and Cybermods, and each of those things is important enough to be in the core rules. I wasn't going to have sixteen playbooks. So the list started Gangs-gangs, Gangs-Holdings (the goon), Gangs-favours (the charity worker)ect, but I didn't fill in all the slots, just enough to represent everything in most four player games.

Then I'll will assign the playbook a GM prompt from the list below. These are things the playbook brings to the table that provide, prompts, hooks, buttons or leashes that help the GM keep the story moving.  They are a good thing to ask provocative questions about.

\begin{itemize}
\item start of session momemtum triggers eg resource shortages
\item Cast of NPCs  eg regular customers
\item catalyser of conflict eg Gunlugger
\item expand map eg Driver
\item Instigate threat eg Vampire Hunter
\item fixed point to defend/orbit eg Hardholder, king \footnote{this goes against classic dungeon crawling where the party must stay together and so anything like this is an impediment to the story.}
\item need for supplies eg Savvyhead
\item supports risky play eg healing
\item countdown urgency eg Vampire
\item broadcast npc motivations eg observant\footnote{broadcasting motivations even to a single player normally means all other players hear it, and it helps bring the world to life a little more for everyone}
\item Mostly obedient entity eg Beast-master.
\end{itemize}

Why go to this effort? It makes writing moves easier. If I know this playbook is about the Detective archetype, that the playbook's primary stat is Street and secondary is Smarts and it engages with the Clue mechanic but not the Guilt or Resources mechanics, and brings the Start of Session prompt for the GM, then I've got a good lead. You can do the same spacing out exercise inside the playbook too. 

Lets say for the inital draft the Detective starts with one Street move and one start of session Clue move, and has options to learn another two Street moves (three total for primary), two Smarts moves and one more Clue move. That's seven moves which is a good set. \footnote{ Often more then Apocalyspe World playbooks get, although they engage heavy specific to playbook mechanics. Dungeon World by contrast, will often hit fifteen or more simpler moves since frequent levelling up is a core part of the Hero's Journey theme.}

What are three characters that this playbook should be able to build? The crumpled, world weary but insightful genius, the hardboiled ex cop simmering with anger, the flawed but charming (and patient) interrogator. The crow, the bull and the spider.

Lets give the Crow archetype the second Clue move (something to do with last minute insight?) and give the Bull and the Spider aspects one move each for Street and Smarts. Of course a player might mix those up and decide to focus on all the Smarts moves, or even, if you allow the option, learn a move from the Femme book and specialise in uncovering details in relationships. Building in a way to allow a move to be learnt from a different playbook goes a long way to allowing players to explore the exact character they are playing. It works as long as the playbooks are spaced enough to prevent `obvious' choices.
 
The trick with this of approach is to constrain it just enough that as designer you can only see one obvious answer, not to give you a headache trying to fit some arbitrary, impossibly difficult combination together. 
When writing some hacks, I might generate each move for the play book randomly, trusting to statistical clumpiness to give each playbook an emergent theme. I'll give examples of that in the next section.
 
 \subsection{Anatomy of a move}
If you write a custom move, roll+WIS
On a 10+ it is thematic, useful and short \\
On a 7-9  it is two of the above but not the third.\\
On a 6- it is a failure. Rely on the GM to try and salvage it.

Sometimes I will have an idea for a particular move for a playbook I will make sure is there. I also like the depth you can get from a wider range of slightly conflicting moves. Randomly rolling them can work for me.

Choose one from each list:

\subsection{Trigger and move shape.}
\begin{itemize}
\item when you do something related to (speciality) Roll for bonus
\item when you do (speciality) roll for/get hold. spend hold for bonus
\item you have the ability to (active power). it counts as base move using (stat)
\item you have (passive power with constant effect)\footnote{by `power' I mean something that changes the fiction - cast light or walk through walls for example}
\item you have a (thing). when applicable it adds two bonuses
\item Straight bonus
\item When you interact with another player, bonus
\end{itemize}

\subsection{Bonus is}
\begin{itemize}
\item bonus of +1 to stat  \footnote{some people disapprove of straight +1 to stat bonus moves, but they have a role for a player who dosen't want more complexity to manage just yet}
\item bonus of reroll a fail
\item power of springboard fiction (eg cast light)
\item power of mechanical fiction (eg healing, clue mechanic)
\item get knowledge 
\item juicy list
\item juicy complications \footnote{normally choose two of three that don't happen.}
\item oddball juicy list \footnote{the favoured move shape of Urban Shadows. Choose three options out of four, where one of the four options is a an oddball.}
\end{itemize}

\subsection{Prompts}
These may look familar from before. They can work well for individual moves to ensure the story moves forward. Prompts may be part of the 7-9 complications, -6 consequences or even +10 bonuses!
\begin{itemize}
\item start of session momentum triggers 
\item cast of NPCs 
\item catalyser of conflict 
\item expand map
\item Instigate threat
\item fixed point to defend/orbit
\item need for supplies
\item supports risky play
\item countdown urgency
\item broadcast npc motivations
\item mostly obedient entity
\end{itemize}

Let's look at that Detective starting Street Move. I roll on the lists above randomly:
When you do something related to (speciality) bonus. Bonus is juicy list. Prompt is Countdown Urgency. I interperet that as:

When you take on a new case, Roll+Street
On a 10+ choose two, On a 7-9 choose two, but one will be lost soon.
\begin{itemize}
\myitem you know you have an informant good for this
\myitem you get a cash down payment up front
\myitemend you see a link to another case
\end{itemize}

Lets roll again for the Clue starting move\footnote{my initial thought before rolling was ``If there is a clue token at stake in this scene, you have a hunch it is available.\\ It is short and engages the theme, but feels like it'd result in a frustrated player as much as a frustrated detective character.''}

You have a (thing). when applicable it adds two bonuses. Bonus of reroll a fail, Prompt of catalyser of conflict . Extra bonus of get knowledge.

You have a stained, run-down office you can barely afford. Choose one thing it has: Fire-escape to alley; loyal secretary called Dolores, filing cabinet with years of neat notes and newspaper clippings.
When you spend the night there, you can anyone ask one question for each Clue token you hold, as though you are back there at the scene. When you act aggressively on the answers, you can reroll the first failed roll. \footnote{it's certainly thematic, might need editing to be a little shorter and tighter, and will need play testing to ensure it's not a spotlight hog}

In the example above I decided which moves had a roll and which didn't. In my experience so far, about one in  ten of the moves won't survive initial playtesting and will be replaced by something else entirely. This is an inital draft to get you to the table as fast as possible. Again, while I like the rolling method to get me beyond the obvious and avoid playbooks with 14 moves keyed to tiny moments of violence. It's a device to help, not block you completely. If a combination feels truely unworkable, or Lady Luck has given you nothing but information gathering moves, adjust them! The random generators can be pushed further too - write down the dozen or so player character archetypes, write down four specialities and two `things' of each archetype, assign each playbook a mix of two archetypes and roll up a speciality or a thing if the move uses it. 

\section{ Development}
Development is the mechanic and rules system that governs how players can develop their characters.\footnote{If you are writing a one shot game, you probably don't need this. }
It is a mechanic that changes often between different PBTA hacks, indicating that no-one has found a solution that everyone else is happy with. It's a perennial favourite topic for discussion, perhaps because it  ties so closely into the hopes, desires and sense of entitlement of players.

\subsection{Simple models}
The simplest model of all is no mechanical development. In that case players start with a `complete' character, and any development occurs in the fiction (acquiring contacts, loot, relationships, exotic pets ect). An example (outside PBTA) is Traveller - where the genre setting of you all playing crusty middle aged `professionals' with a ship sized mortgage to feed means that `static' character skill sets fit the genre. Using this method  focuses player development urges into the fiction and reduces mechanical complexity. It reduces how complex a playbook can be be (since you don't want to overwhelm someone with a dozen things to memorise right at the start) and dosen't give a player an escape route to develop the playbook more into their style. 

The next simplest is session levelling, where you get an upgrade at the end of every session or every two sessions. This can feel arbitrary and unsatisfying because you don't have to `work for it'. It is easy to implement, and to be honest most games try and calibrate for this speed of development anyway. 

Milestone levelling is another option, where you `level up' after hitting some longer term arc/project. The Savvyhead in Apocalypse World kind of does this by default - as they complete projects, the MC will often give them a custom move that ties into that gear. The limits with this method is that is suits a team game with a shared goal much more then players at cross purposes. You really don't want to put the gm in a position where they have to make calls that will let one player milestone up and block another. That kind of bleed does not feel like a good thing at the end of the session.  I think Nightwitches uses it as it suits the mission based, team based nature of the setting.

\subsection{XP models}
Most PBTA games use XP points, with normally a few different things feeding into your XP total. I am fairly neutral on these. Pick whatever drives thematic behaviour.

Dungeon World and Monster of the Week gives you XP on a `failed' roll.
They both give you `top up' xp if you can successfully answer three thematic questions at the end of session\footnote{ie did you find treasure, explore the world, learn something new ect}.
The Sprawl plays against the team dynamic, and offers XP when you follow personal directives that put the group at risk.
Apocalypse Word gives you XP when you max out and reset your relationship stat with someone. Dungeon World does something similar with bonds. 
World of Adventure has adopted Keys - which give small XP when triggered and big XP when completed and replaced\footnote{Ie Key of the Merchant \\ Gain 1Xp when you negotiate a trade that makes profit by the end of session \\ Buyoff: 3xp and change key When you refuse to part with something at any price}
Apocalypse World also gives you XP when you roll with a highlighted stat. Twist is that other players choose the highlight as a hint what they'd like to see at the table. 

\section{Scenario Moves}
This is a way of bulking out the basic moves to keep them from getting boring, without asking your players to memorise a ton of moves right at the very start of the game.
The sprawl uses these to cover less commonly occurring, but thematically very important situations in the cyberpunk stories - getting the job, dying and getting cybermods. Sometimes in that order.
Apocalypse World uses them to cover complicated motorcade warfare without relying on a single basic move all the time. 
Dungeon World could perhaps benefit from a `Scouting mission' scenario move set covering falling, sneaking, climbing, drowning ect.

\section{Inserts}
These are a different way of handling complex `modules' of rules that are not moves. Dungeon World uses them for Pet Familiars. Apocalypse World uses them for gangs and vehicles. They are literally a self contained page of rules that you can append to a playbook to handle something special. 
You know the Driver starts with a special vehicle, but it is easy to see how many player characters may acquire one during play and need to insert the rules for their whiskey powered motorbike into their play book.

If you want to free a specific thing from a playbook archetype, or in playtesting people keep insisting on owning a particular thing/gang/pet -  an insert may be the way forward! 
Like a playbook, I suggest a mix of fixed options and thematic rules for name, appearance, personality traits or predictable behaviour when the player rolls a fail. A few explicit ways for the player to upgrade the insert are also good. 


\section{Tags}.
WIP
\section{Theme Deconstruction}

\subsection{balance}
+ GM Principle: meaningful player character decisions shift the balance
+ GM Move: Worst imbalance reduces but new imbalance develops.	
Possible mechanic: split playbook experience and moves into balance/imbalance tracks

\subsection{belief}
+ GM Principle: not all belief is true. sometimes the players are right
+ GM Move trade loss for benefit (in accordance with belief)
+ Possible mechanic: ritual elements to reinforce character will or add details to world

\subsection{change}
+ GM Principle: What has changed since the player's last visited. What hasn't?	
+ GM Move: Change one constraint on the situation, people or landscape
+ Possible Mechanic: push your luck for directed change of landscape or yourself ?

\subsection{comfort}
+ GM Principle: agree with players what comfort means. check in occasionally.
+GM Move: Throw a feast, party or concert
+ Possible Mechanic: meet npcs comfort level to gain ally

\subsection{destruction }
+ GM Principle: First consider destroying npcs and your MC ideas	
+ GM Move: "trade harm for harm (as established) or deal harm (as established)"	
+ Possible mechanic: Detailed harm tags (messy, knockback, seige ect)


\subsection{corruption}
+GM Principle: corruption requires the player's character to feel guilt over their actions	
+GM Move:  offer a conflict of interest. both good	
+ Possible Mechanic: Corruption moves that can be activated on a spefific trigger

\subsection{family}
+GM Principle: every parent was once a child
+GM Move: 	announce a flashback
+ Possible Mechanic: 	inheritance of soft moves

\subsection{freedom}
+GM Principle:  There is always another place to run to
+GM Move:  ask a silent player to describe location ahead.
+ Possible Mechanic: expansion of vehicles into mobile bases 

\subsection{friendship}
+GM Principle: friendship spreads through interlocking circles cemented by favours
+GM Move: an npc calls in a favour. reputation cost to say no 
+ Possible Mechanic: do favours, receive debts. call in for favours
		
\subsection{glory}
+GM Principle: wherever they go, PCs should hear of great adventurers. sometimes, it might even be them
+GM Move: create a competition for the players
+ Possible Mechanic: gained by beating more difficult challenges and spent to create more difficult challenges  

\subsection{honour}
+GM Principle: always ask if an npc action is honourable. justify it
+GM Move: deal harm to players honour, justified or sneaky
+ Possible Mechanic: honour only applies to certain ranks. it means that oaths carry weight but only for those ranks.

\subsection{hunger}
+GM Principle:  always describe food, fatness, health first
+GM Move:   reveal an upcoming scarcity
+ Possible Mechanic: needs and consequences tag system

\subsection{justice}
+GM Principle: What typifies this situation: the scales, the blindfold or the sword?	
+GM Move: trade judgemet, compassion or violence (as established)
+ Possible Mechanic: ask the abyss "is this just?" Mark bonuses at end of session 

\subsection{knowledge }
+GM Principle: knowledge is the lever that player characters can use to magnify their actions	
+GM Move: trade knowledge for knowledge (as established)
+ Possible Mechanic: cash in knowledge for bonuses on quadratic scale

\subsection{law/tradition}
+GM Principle: everybody, everything has a couple of freely known behaviours
+GM Move: Invoke tradition to create a new threat
+ Possible Mechanic: two polar stats to make quadrant block between four traditions. 

\subsection{love}
+GM Principle:  love is steadily rising level of intimacy
+GM Move: spotlight a connection or triangle
+ Possible Mechanic: intimacy moves: 	 "i'd like to X, may I?", "yes but"

\subsection{pain}
+GM Principle: Scars are not just physical. Damage is not just permanent
+GM Move: deal pain, as established
+ Possible Mechanic: memory/pain as addiction 

\subsection{revenge}

+GM Principle: NPCs will always be avenged by someone
+GM Move:  offer vendetta, offer ritual peace at cost
+ Possible Mechanic: detailed followers and bonds rules. fail to avenge one, loose others due to disgrace.

\subsection{unity}
+GM Principle: always introduce someone by their unit/faction/clan before any other detail	
+GM Move: create player-npc-unit triangle over a resource 	
+ Possible Mechanic: Faction tags on all players and NPcs that can be used as PC move triggers


\subsection{survival}
+GM Principle: respond with fuckery and intermittent rewards
+GM Move:  offer something, but make them roll the dice
+ Possible Mechanic: Crafting guidelines for making better stuff from crappy parts

\subsection{violence}
+GM Principle: Sometimes the direct solution is the correct one
+GM Move: 	add violence not directed at pcs
+ Possible Mechanic: conflict escalation mechanic. disagreement to posturing to threats to violence 

\subsection{wealth}
+GM Principle: currency is liquid. where does it flow?
+GM Move: replace a source of wealth with another	
+ possible Mechanic: investment schemes as missions 


%\section{Page Layout}\label{sec:page-layout}
%\subsection{Headings}\label{sec:headings}
%This style provides \textsc{a}- and \textsc{b}-heads (that is,
%\Verb|\section| and \Verb|\subsection|), demonstrated above.

%The Tufte-\LaTeX\ classes will emit an error if you try to use
%\linebreak\Verb|\subsubsection| and smaller headings.

% let's start a new thought -- a new section
%\newthought{In his later books},\cite{Tufte2006} Tufte

\bibliography{Skeleton_world}
\bibliographystyle{plainnat}



\end{document}
